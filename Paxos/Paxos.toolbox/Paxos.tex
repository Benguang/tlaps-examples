\batchmode %% Suppresses most terminal output.
\documentclass{article}
\usepackage{color}
\definecolor{boxshade}{gray}{0.85}
\setlength{\textwidth}{360pt}
\setlength{\textheight}{541pt}
\usepackage{latexsym}
\usepackage{ifthen}
% \usepackage{color}
%%%%%%%%%%%%%%%%%%%%%%%%%%%%%%%%%%%%%%%%%%%%%%%%%%%%%%%%%%%%%%%%%%%%%%%%%%%%%
% SWITCHES                                                                  %
%%%%%%%%%%%%%%%%%%%%%%%%%%%%%%%%%%%%%%%%%%%%%%%%%%%%%%%%%%%%%%%%%%%%%%%%%%%%%
\newboolean{shading} 
\setboolean{shading}{false}
\makeatletter
 %% this is needed only when inserted into the file, not when
 %% used as a package file.
%%%%%%%%%%%%%%%%%%%%%%%%%%%%%%%%%%%%%%%%%%%%%%%%%%%%%%%%%%%%%%%%%%%%%%%%%%%%%
%                                                                           %
% DEFINITIONS OF SYMBOL-PRODUCING COMMANDS                                  %
%                                                                           %
%    TLA+      LaTeX                                                        %
%    symbol    command                                                      %
%    ------    -------                                                      %
%    =>        \implies                                                     %
%    <:        \ltcolon                                                     %
%    :>        \colongt                                                     %
%    ==        \defeq                                                       %
%    ..        \dotdot                                                      %
%    ::        \coloncolon                                                  %
%    =|        \eqdash                                                      %
%    ++        \pp                                                          %
%    --        \mm                                                          %
%    **        \stst                                                        %
%    //        \slsl                                                        %
%    ^         \ct                                                          %
%    \A        \A                                                           %
%    \E        \E                                                           %
%    \AA       \AA                                                          %
%    \EE       \EE                                                          %
%%%%%%%%%%%%%%%%%%%%%%%%%%%%%%%%%%%%%%%%%%%%%%%%%%%%%%%%%%%%%%%%%%%%%%%%%%%%%
\newlength{\symlength}
\newcommand{\implies}{\Rightarrow}
\newcommand{\ltcolon}{\mathrel{<\!\!\mbox{:}}}
\newcommand{\colongt}{\mathrel{\!\mbox{:}\!\!>}}
\newcommand{\defeq}{\;\mathrel{\smash   %% keep this symbol from being too tall
    {{\stackrel{\scriptscriptstyle\Delta}{=}}}}\;}
\newcommand{\dotdot}{\mathrel{\ldotp\ldotp}}
\newcommand{\coloncolon}{\mathrel{::\;}}
\newcommand{\eqdash}{\mathrel = \joinrel \hspace{-.28em}|}
\newcommand{\pp}{\mathbin{++}}
\newcommand{\mm}{\mathbin{--}}
\newcommand{\stst}{*\!*}
\newcommand{\slsl}{/\!/}
\newcommand{\ct}{\hat{\hspace{.4em}}}
\newcommand{\A}{\forall}
\newcommand{\E}{\exists}
\renewcommand{\AA}{\makebox{$\raisebox{.05em}{\makebox[0pt][l]{%
   $\forall\hspace{-.517em}\forall\hspace{-.517em}\forall$}}%
   \forall\hspace{-.517em}\forall \hspace{-.517em}\forall\,$}}
\newcommand{\EE}{\makebox{$\raisebox{.05em}{\makebox[0pt][l]{%
   $\exists\hspace{-.517em}\exists\hspace{-.517em}\exists$}}%
   \exists\hspace{-.517em}\exists\hspace{-.517em}\exists\,$}}
\newcommand{\whileop}{\.{\stackrel
  {\mbox{\raisebox{-.3em}[0pt][0pt]{$\scriptscriptstyle+\;\,$}}}%
  {-\hspace{-.16em}\triangleright}}}

% Commands are defined to produce the upper-case keywords.
% Note that some have space after them.
\newcommand{\ASSUME}{\textsc{assume }}
\newcommand{\ASSUMPTION}{\textsc{assumption }}
\newcommand{\AXIOM}{\textsc{axiom }}
\newcommand{\BOOLEAN}{\textsc{boolean }}
\newcommand{\CASE}{\textsc{case }}
\newcommand{\CONSTANT}{\textsc{constant }}
\newcommand{\CONSTANTS}{\textsc{constants }}
\newcommand{\ELSE}{\settowidth{\symlength}{\THEN}%
   \makebox[\symlength][l]{\textsc{ else}}}
\newcommand{\EXCEPT}{\textsc{ except }}
\newcommand{\EXTENDS}{\textsc{extends }}
\newcommand{\FALSE}{\textsc{false}}
\newcommand{\IF}{\textsc{if }}
\newcommand{\IN}{\settowidth{\symlength}{\LET}%
   \makebox[\symlength][l]{\textsc{in}}}
\newcommand{\INSTANCE}{\textsc{instance }}
\newcommand{\LET}{\textsc{let }}
\newcommand{\LOCAL}{\textsc{local }}
\newcommand{\MODULE}{\textsc{module }}
\newcommand{\OTHER}{\textsc{other }}
\newcommand{\STRING}{\textsc{string}}
\newcommand{\THEN}{\textsc{ then }}
\newcommand{\THEOREM}{\textsc{theorem }}
\newcommand{\LEMMA}{\textsc{lemma }}
\newcommand{\PROPOSITION}{\textsc{proposition }}
\newcommand{\COROLLARY}{\textsc{corollary }}
\newcommand{\TRUE}{\textsc{true}}
\newcommand{\VARIABLE}{\textsc{variable }}
\newcommand{\VARIABLES}{\textsc{variables }}
\newcommand{\WITH}{\textsc{ with }}
\newcommand{\WF}{\textrm{WF}}
\newcommand{\SF}{\textrm{SF}}
\newcommand{\CHOOSE}{\textsc{choose }}
\newcommand{\ENABLED}{\textsc{enabled }}
\newcommand{\UNCHANGED}{\textsc{unchanged }}
\newcommand{\SUBSET}{\textsc{subset }}
\newcommand{\UNION}{\textsc{union }}
\newcommand{\DOMAIN}{\textsc{domain }}
% Added for tla2tex
\newcommand{\BY}{\textsc{by }}
\newcommand{\OBVIOUS}{\textsc{obvious }}
\newcommand{\HAVE}{\textsc{have }}
\newcommand{\QED}{\textsc{qed }}
\newcommand{\TAKE}{\textsc{take }}
\newcommand{\DEF}{\textsc{ def }}
\newcommand{\HIDE}{\textsc{hide }}
\newcommand{\RECURSIVE}{\textsc{recursive }}
\newcommand{\USE}{\textsc{use }}
\newcommand{\DEFINE}{\textsc{define }}
\newcommand{\PROOF}{\textsc{proof }}
\newcommand{\WITNESS}{\textsc{witness }}
\newcommand{\PICK}{\textsc{pick }}
\newcommand{\DEFS}{\textsc{defs }}
\newcommand{\PROVE}{\settowidth{\symlength}{\ASSUME}%
   \makebox[\symlength][l]{\textsc{prove}}\@s{-4.1}}%
  %% The \@s{-4.1) is a kludge added on 24 Oct 2009 [happy birthday, Ellen]
  %% so the correct alignment occurs if the user types
  %%   ASSUME X
  %%   PROVE  Y
  %% because it cancels the extra 4.1 pts added because of the 
  %% extra space after the PROVE.  This seems to works OK.
  %% However, the 4.1 equals Parameters.LaTeXLeftSpace(1) and
  %% should be changed if that method ever changes.
\newcommand{\SUFFICES}{\textsc{suffices }}
\newcommand{\NEW}{\textsc{new }}
\newcommand{\LAMBDA}{\textsc{lambda }}
\newcommand{\STATE}{\textsc{state }}
\newcommand{\ACTION}{\textsc{action }}
\newcommand{\TEMPORAL}{\textsc{temporal }}
\newcommand{\ONLY}{\textsc{only }}              %% added by LL on 2 Oct 2009
\newcommand{\OMITTED}{\textsc{omitted }}        %% added by LL on 31 Oct 2009
\newcommand{\@pfstepnum}[2]{\ensuremath{\langle#1\rangle}\textrm{#2}}
\newcommand{\bang}{\@s{1}\mbox{\small !}\@s{1}}
%% We should format || differently in PlusCal code than in TLA+ formulas.
\newcommand{\p@barbar}{\ifpcalsymbols
   \,\,\rule[-.25em]{.075em}{1em}\hspace*{.2em}\rule[-.25em]{.075em}{1em}\,\,%
   \else \,||\,\fi}
%% PlusCal keywords
\newcommand{\p@fair}{\textbf{fair }}
\newcommand{\p@semicolon}{\textbf{\,; }}
\newcommand{\p@algorithm}{\textbf{algorithm }}
\newcommand{\p@mmfair}{\textbf{-{}-fair }}
\newcommand{\p@mmalgorithm}{\textbf{-{}-algorithm }}
\newcommand{\p@assert}{\textbf{assert }}
\newcommand{\p@await}{\textbf{await }}
\newcommand{\p@begin}{\textbf{begin }}
\newcommand{\p@end}{\textbf{end }}
\newcommand{\p@call}{\textbf{call }}
\newcommand{\p@define}{\textbf{define }}
\newcommand{\p@do}{\textbf{ do }}
\newcommand{\p@either}{\textbf{either }}
\newcommand{\p@or}{\textbf{or }}
\newcommand{\p@goto}{\textbf{goto }}
\newcommand{\p@if}{\textbf{if }}
\newcommand{\p@then}{\,\,\textbf{then }}
\newcommand{\p@else}{\ifcsyntax \textbf{else } \else \,\,\textbf{else }\fi}
\newcommand{\p@elsif}{\,\,\textbf{elsif }}
\newcommand{\p@macro}{\textbf{macro }}
\newcommand{\p@print}{\textbf{print }}
\newcommand{\p@procedure}{\textbf{procedure }}
\newcommand{\p@process}{\textbf{process }}
\newcommand{\p@return}{\textbf{return}}
\newcommand{\p@skip}{\textbf{skip}}
\newcommand{\p@variable}{\textbf{variable }}
\newcommand{\p@variables}{\textbf{variables }}
\newcommand{\p@while}{\textbf{while }}
\newcommand{\p@when}{\textbf{when }}
\newcommand{\p@with}{\textbf{with }}
\newcommand{\p@lparen}{\textbf{(\,\,}}
\newcommand{\p@rparen}{\textbf{\,\,) }}   
\newcommand{\p@lbrace}{\textbf{\{\,\,}}   
\newcommand{\p@rbrace}{\textbf{\,\,\} }}

%%%%%%%%%%%%%%%%%%%%%%%%%%%%%%%%%%%%%%%%%%%%%%%%%%%%%%%%%
% REDEFINE STANDARD COMMANDS TO MAKE THEM FORMAT BETTER %
%                                                       %
% We redefine \in and \notin                            %
%%%%%%%%%%%%%%%%%%%%%%%%%%%%%%%%%%%%%%%%%%%%%%%%%%%%%%%%%
\renewcommand{\_}{\rule{.4em}{.06em}\hspace{.05em}}
\newlength{\equalswidth}
\let\oldin=\in
\let\oldnotin=\notin
\renewcommand{\in}{%
   {\settowidth{\equalswidth}{$\.{=}$}\makebox[\equalswidth][c]{$\oldin$}}}
\renewcommand{\notin}{%
   {\settowidth{\equalswidth}{$\.{=}$}\makebox[\equalswidth]{$\oldnotin$}}}


%%%%%%%%%%%%%%%%%%%%%%%%%%%%%%%%%%%%%%%%%%%%%%%%%%%%
%                                                  %
% HORIZONTAL BARS:                                 %
%                                                  %
%   \moduleLeftDash    |~~~~~~~~~~                 %
%   \moduleRightDash    ~~~~~~~~~~|                %
%   \midbar            |----------|                %
%   \bottombar         |__________|                %
%%%%%%%%%%%%%%%%%%%%%%%%%%%%%%%%%%%%%%%%%%%%%%%%%%%%
\newlength{\charwidth}\settowidth{\charwidth}{{\small\tt M}}
\newlength{\boxrulewd}\setlength{\boxrulewd}{.4pt}
\newlength{\boxlineht}\setlength{\boxlineht}{.5\baselineskip}
\newcommand{\boxsep}{\charwidth}
\newlength{\boxruleht}\setlength{\boxruleht}{.5ex}
\newlength{\boxruledp}\setlength{\boxruledp}{-\boxruleht}
\addtolength{\boxruledp}{\boxrulewd}
\newcommand{\boxrule}{\leaders\hrule height \boxruleht depth \boxruledp
                      \hfill\mbox{}}
\newcommand{\@computerule}{%
  \setlength{\boxruleht}{.5ex}%
  \setlength{\boxruledp}{-\boxruleht}%
  \addtolength{\boxruledp}{\boxrulewd}}

\newcommand{\bottombar}{\hspace{-\boxsep}%
  \raisebox{-\boxrulewd}[0pt][0pt]{\rule[.5ex]{\boxrulewd}{\boxlineht}}%
  \boxrule
  \raisebox{-\boxrulewd}[0pt][0pt]{%
      \rule[.5ex]{\boxrulewd}{\boxlineht}}\hspace{-\boxsep}\vspace{0pt}}

\newcommand{\moduleLeftDash}%
   {\hspace*{-\boxsep}%
     \raisebox{-\boxlineht}[0pt][0pt]{\rule[.5ex]{\boxrulewd
               }{\boxlineht}}%
    \boxrule\hspace*{.4em }}

\newcommand{\moduleRightDash}%
    {\hspace*{.4em}\boxrule
    \raisebox{-\boxlineht}[0pt][0pt]{\rule[.5ex]{\boxrulewd
               }{\boxlineht}}\hspace{-\boxsep}}%\vspace{.2em}

\newcommand{\midbar}{\hspace{-\boxsep}\raisebox{-.5\boxlineht}[0pt][0pt]{%
   \rule[.5ex]{\boxrulewd}{\boxlineht}}\boxrule\raisebox{-.5\boxlineht%
   }[0pt][0pt]{\rule[.5ex]{\boxrulewd}{\boxlineht}}\hspace{-\boxsep}}

%%%%%%%%%%%%%%%%%%%%%%%%%%%%%%%%%%%%%%%%%%%%%%%%%%%%%%%%%%%%%%%%%%%%%%%%%%%%%
% FORMATING COMMANDS                                                        %
%%%%%%%%%%%%%%%%%%%%%%%%%%%%%%%%%%%%%%%%%%%%%%%%%%%%%%%%%%%%%%%%%%%%%%%%%%%%%

%%%%%%%%%%%%%%%%%%%%%%%%%%%%%%%%%%%%%%%%%%%%%%%%%%%%%%%%%%%%%%%%%%%%%%%%%%%%%
% PLUSCAL SHADING                                                           %
%%%%%%%%%%%%%%%%%%%%%%%%%%%%%%%%%%%%%%%%%%%%%%%%%%%%%%%%%%%%%%%%%%%%%%%%%%%%%

% The TeX pcalshading switch is set on to cause PlusCal shading to be
% performed.  This changes the behavior of the following commands and
% environments to cause full-width shading to be performed on all lines.
% 
%   \tstrut \@x cpar mcom \@pvspace
% 
% The TeX pcalsymbols switch is turned on when typesetting a PlusCal algorithm,
% whether or not shading is being performed.  It causes symbols (other than
% parentheses and braces and PlusCal-only keywords) that should be typeset
% differently depending on whether they are in an algorithm to be typeset
% appropriately.  Currently, the only such symbol is "||".
%
% The TeX csyntax switch is turned on when typesetting a PlusCal algorithm in
% c-syntax.  This allows symbols to be format differently in the two syntaxes.
% The "else" keyword is the only one that is.

\newif\ifpcalshading \pcalshadingfalse
\newif\ifpcalsymbols \pcalsymbolsfalse
\newif\ifcsyntax     \csyntaxtrue

% The \@pvspace command makes a vertical space.  It uses \vspace
% except with \ifpcalshading, in which case it sets \pvcalvspace
% and the space is added by a following \@x command.
%
\newlength{\pcalvspace}\setlength{\pcalvspace}{0pt}%
\newcommand{\@pvspace}[1]{%
  \ifpcalshading
     \par\global\setlength{\pcalvspace}{#1}%
  \else
     \par\vspace{#1}%
  \fi
}

% The lcom environment was changed to set \lcomindent equal to
% the indentation it produces.  This length is used by the
% cpar environment to make shading extend for the full width
% of the line.  This assumes that lcom environments are not
% nested.  I hope TLATeX does not nest them.
%
\newlength{\lcomindent}%
\setlength{\lcomindent}{0pt}%

%\tstrut: A strut to produce inter-paragraph space in a comment.
%\rstrut: A strut to extend the bottom of a one-line comment so
%         there's no break in the shading between comments on 
%         successive lines.
\newcommand\tstrut%
  {\raisebox{\vshadelen}{\raisebox{-.25em}{\rule{0pt}{1.15em}}}%
   \global\setlength{\vshadelen}{0pt}}
\newcommand\rstrut{\raisebox{-.25em}{\rule{0pt}{1.15em}}%
 \global\setlength{\vshadelen}{0pt}}


% \.{op} formats operator op in math mode with empty boxes on either side.
% Used because TeX otherwise vary the amount of space it leaves around op.
\renewcommand{\.}[1]{\ensuremath{\mbox{}#1\mbox{}}}

% \@s{n} produces an n-point space
\newcommand{\@s}[1]{\hspace{#1pt}}           

% \@x{txt} starts a specification line in the beginning with txt
% in the final LaTeX source.
\newlength{\@xlen}
\newcommand\xtstrut%
  {\setlength{\@xlen}{1.05em}%
   \addtolength{\@xlen}{\pcalvspace}%
    \raisebox{\vshadelen}{\raisebox{-.25em}{\rule{0pt}{\@xlen}}}%
   \global\setlength{\vshadelen}{0pt}%
   \global\setlength{\pcalvspace}{0pt}}

\newcommand{\@x}[1]{\par
  \ifpcalshading
  \makebox[0pt][l]{\shadebox{\xtstrut\hspace*{\textwidth}}}%
  \fi
  \mbox{$\mbox{}#1\mbox{}$}}  

% \@xx{txt} continues a specification line with the text txt.
\newcommand{\@xx}[1]{\mbox{$\mbox{}#1\mbox{}$}}  

% \@y{cmt} produces a one-line comment.
\newcommand{\@y}[1]{\mbox{\footnotesize\hspace{.65em}%
  \ifthenelse{\boolean{shading}}{%
      \shadebox{#1\hspace{-\the\lastskip}\rstrut}}%
               {#1\hspace{-\the\lastskip}\rstrut}}}

% \@z{cmt} produces a zero-width one-line comment.
\newcommand{\@z}[1]{\makebox[0pt][l]{\footnotesize
  \ifthenelse{\boolean{shading}}{%
      \shadebox{#1\hspace{-\the\lastskip}\rstrut}}%
               {#1\hspace{-\the\lastskip}\rstrut}}}


% \@w{str} produces the TLA+ string "str".
\newcommand{\@w}[1]{\textsf{``{#1}''}}             


%%%%%%%%%%%%%%%%%%%%%%%%%%%%%%%%%%%%%%%%%%%%%%%%%%%%%%%%%%%%%%%%%%%%%%%%%%%%%
% SHADING                                                                   %
%%%%%%%%%%%%%%%%%%%%%%%%%%%%%%%%%%%%%%%%%%%%%%%%%%%%%%%%%%%%%%%%%%%%%%%%%%%%%
\def\graymargin{1}
  % The number of points of margin in the shaded box.

% \definecolor{boxshade}{gray}{.85}
% Defines the darkness of the shading: 1 = white, 0 = black
% Added by TLATeX only if needed.

% \shadebox{txt} puts txt in a shaded box.
\newlength{\templena}
\newlength{\templenb}
\newsavebox{\tempboxa}
\newcommand{\shadebox}[1]{{\setlength{\fboxsep}{\graymargin pt}%
     \savebox{\tempboxa}{#1}%
     \settoheight{\templena}{\usebox{\tempboxa}}%
     \settodepth{\templenb}{\usebox{\tempboxa}}%
     \hspace*{-\fboxsep}\raisebox{0pt}[\templena][\templenb]%
        {\colorbox{boxshade}{\usebox{\tempboxa}}}\hspace*{-\fboxsep}}}

% \vshade{n} makes an n-point inter-paragraph space, with
%  shading if the `shading' flag is true.
\newlength{\vshadelen}
\setlength{\vshadelen}{0pt}
\newcommand{\vshade}[1]{\ifthenelse{\boolean{shading}}%
   {\global\setlength{\vshadelen}{#1pt}}%
   {\vspace{#1pt}}}

\newlength{\boxwidth}
\newlength{\multicommentdepth}

%%%%%%%%%%%%%%%%%%%%%%%%%%%%%%%%%%%%%%%%%%%%%%%%%%%%%%%%%%%%%%%%%%%%%%%%%%%%%
% THE cpar ENVIRONMENT                                                      %
% ^^^^^^^^^^^^^^^^^^^^                                                      %
% The LaTeX input                                                           %
%                                                                           %
%   \begin{cpar}{pop}{nest}{isLabel}{d}{e}{arg6}                            %
%     XXXXXXXXXXXXXXX                                                       %
%     XXXXXXXXXXXXXXX                                                       %
%     XXXXXXXXXXXXXXX                                                       %
%   \end{cpar}                                                              %
%                                                                           %
% produces one of two possible results.  If isLabel is the letter "T",      %
% it produces the following, where [label] is the result of typesetting     %
% arg6 in an LR box, and d is is a number representing a distance in        %
% points.                                                                   %
%                                                                           %
%   prevailing |<-- d -->[label]<- e ->XXXXXXXXXXXXXXX                      %
%         left |                       XXXXXXXXXXXXXXX                      %
%       margin |                       XXXXXXXXXXXXXXX                      %
%                                                                           %
% If isLabel is the letter "F", then it produces                            %
%                                                                           %
%   prevailing |<-- d -->XXXXXXXXXXXXXXXXXXXXXXX                            %
%         left |         <- e ->XXXXXXXXXXXXXXXX                            %
%       margin |                XXXXXXXXXXXXXXXX                            %
%                                                                           %
% where d and e are numbers representing distances in points.               %
%                                                                           %
% The prevailing left margin is the one in effect before the most recent    %
% pop (argument 1) cpar environments with "T" as the nest argument, where   %
% pop is a number \geq 0.                                                   %
%                                                                           %
% If the nest argument is the letter "T", then the prevailing left          %
% margin is moved to the left of the second (and following) lines of        %
% X's.  Otherwise, the prevailing left margin is left unchanged.            %
%                                                                           %
% An \unnest{n} command moves the prevailing left margin to where it was    %
% before the most recent n cpar environments with "T" as the nesting        %
% argument.                                                                 %
%                                                                           %
% The environment leaves no vertical space above or below it, or between    %
% its paragraphs.  (TLATeX inserts the proper amount of vertical space.)    %
%%%%%%%%%%%%%%%%%%%%%%%%%%%%%%%%%%%%%%%%%%%%%%%%%%%%%%%%%%%%%%%%%%%%%%%%%%%%%

\newcounter{pardepth}
\setcounter{pardepth}{0}

% \setgmargin{txt} defines \gmarginN to be txt, where N is \roman{pardepth}.
% \thegmargin equals \gmarginN, where N is \roman{pardepth}.
\newcommand{\setgmargin}[1]{%
  \expandafter\xdef\csname gmargin\roman{pardepth}\endcsname{#1}}
\newcommand{\thegmargin}{\csname gmargin\roman{pardepth}\endcsname}
\newcommand{\gmargin}{0pt}

\newsavebox{\tempsbox}

\newlength{\@cparht}
\newlength{\@cpardp}
\newenvironment{cpar}[6]{%
  \addtocounter{pardepth}{-#1}%
  \ifthenelse{\boolean{shading}}{\par\begin{lrbox}{\tempsbox}%
                                 \begin{minipage}[t]{\linewidth}}{}%
  \begin{list}{}{%
     \edef\temp{\thegmargin}
     \ifthenelse{\equal{#3}{T}}%
       {\settowidth{\leftmargin}{\hspace{\temp}\footnotesize #6\hspace{#5pt}}%
        \addtolength{\leftmargin}{#4pt}}%
       {\setlength{\leftmargin}{#4pt}%
        \addtolength{\leftmargin}{#5pt}%
        \addtolength{\leftmargin}{\temp}%
        \setlength{\itemindent}{-#5pt}}%
      \ifthenelse{\equal{#2}{T}}{\addtocounter{pardepth}{1}%
                                 \setgmargin{\the\leftmargin}}{}%
      \setlength{\labelwidth}{0pt}%
      \setlength{\labelsep}{0pt}%
      \setlength{\itemindent}{-\leftmargin}%
      \setlength{\topsep}{0pt}%
      \setlength{\parsep}{0pt}%
      \setlength{\partopsep}{0pt}%
      \setlength{\parskip}{0pt}%
      \setlength{\itemsep}{0pt}
      \setlength{\itemindent}{#4pt}%
      \addtolength{\itemindent}{-\leftmargin}}%
   \ifthenelse{\equal{#3}{T}}%
      {\item[\tstrut\footnotesize \hspace{\temp}{#6}\hspace{#5pt}]
        }%
      {\item[\tstrut\hspace{\temp}]%
         }%
   \footnotesize}
 {\hspace{-\the\lastskip}\tstrut
 \end{list}%
  \ifthenelse{\boolean{shading}}%
          {\end{minipage}%
           \end{lrbox}%
           \ifpcalshading
             \setlength{\@cparht}{\ht\tempsbox}%
             \setlength{\@cpardp}{\dp\tempsbox}%
             \addtolength{\@cparht}{.15em}%
             \addtolength{\@cpardp}{.2em}%
             \addtolength{\@cparht}{\@cpardp}%
            % I don't know what's going on here.  I want to add a
            % \pcalvspace high shaded line, but I don't know how to
            % do it.  A little trial and error shows that the following
            % does a reasonable job approximating that, eliminating
            % the line if \pcalvspace is small.
            \addtolength{\@cparht}{\pcalvspace}%
             \ifdim \pcalvspace > .8em
               \addtolength{\pcalvspace}{-.2em}%
               \hspace*{-\lcomindent}%
               \shadebox{\rule{0pt}{\pcalvspace}\hspace*{\textwidth}}\par
               \global\setlength{\pcalvspace}{0pt}%
               \fi
             \hspace*{-\lcomindent}%
             \makebox[0pt][l]{\raisebox{-\@cpardp}[0pt][0pt]{%
                 \shadebox{\rule{0pt}{\@cparht}\hspace*{\textwidth}}}}%
             \hspace*{\lcomindent}\usebox{\tempsbox}%
             \par
           \else
             \shadebox{\usebox{\tempsbox}}\par
           \fi}%
           {}%
  }

%%%%%%%%%%%%%%%%%%%%%%%%%%%%%%%%%%%%%%%%%%%%%%%%%%%%%%%%%%%%%%%%%%%%%%%%%%%%%%
% THE ppar ENVIRONMENT                                                       %
% ^^^^^^^^^^^^^^^^^^^^                                                       %
% The environment                                                            %
%                                                                            %
%   \begin{ppar} ... \end{ppar}                                              %
%                                                                            %
% is equivalent to                                                           %
%                                                                            %
%   \begin{cpar}{0}{F}{F}{0}{0}{} ... \end{cpar}                             %
%                                                                            %
% The environment is put around each line of the output for a PlusCal        %
% algorithm.                                                                 %
%%%%%%%%%%%%%%%%%%%%%%%%%%%%%%%%%%%%%%%%%%%%%%%%%%%%%%%%%%%%%%%%%%%%%%%%%%%%%%
%\newenvironment{ppar}{%
%  \ifthenelse{\boolean{shading}}{\par\begin{lrbox}{\tempsbox}%
%                                 \begin{minipage}[t]{\linewidth}}{}%
%  \begin{list}{}{%
%     \edef\temp{\thegmargin}
%        \setlength{\leftmargin}{0pt}%
%        \addtolength{\leftmargin}{\temp}%
%        \setlength{\itemindent}{0pt}%
%      \setlength{\labelwidth}{0pt}%
%      \setlength{\labelsep}{0pt}%
%      \setlength{\itemindent}{-\leftmargin}%
%      \setlength{\topsep}{0pt}%
%      \setlength{\parsep}{0pt}%
%      \setlength{\partopsep}{0pt}%
%      \setlength{\parskip}{0pt}%
%      \setlength{\itemsep}{0pt}
%      \setlength{\itemindent}{0pt}%
%      \addtolength{\itemindent}{-\leftmargin}}%
%      \item[\tstrut\hspace{\temp}]}%
% {\hspace{-\the\lastskip}\tstrut
% \end{list}%
%  \ifthenelse{\boolean{shading}}{\end{minipage}  
%                                 \end{lrbox}%
%                                 \shadebox{\usebox{\tempsbox}}\par}{}%
%  }

 %%% TESTING
 \newcommand{\xtest}[1]{\par
 \makebox[0pt][l]{\shadebox{\xtstrut\hspace*{\textwidth}}}%
 \mbox{$\mbox{}#1\mbox{}$}} 

% \newcommand{\xxtest}[1]{\par
% \makebox[0pt][l]{\shadebox{\xtstrut{#1}\hspace*{\textwidth}}}%
% \mbox{$\mbox{}#1\mbox{}$}} 

%\newlength{\pcalvspace}
%\setlength{\pcalvspace}{0pt}
% \newlength{\xxtestlen}
% \setlength{\xxtestlen}{0pt}
% \newcommand\xtstrut%
%   {\setlength{\xxtestlen}{1.15em}%
%    \addtolength{\xxtestlen}{\pcalvspace}%
%     \raisebox{\vshadelen}{\raisebox{-.25em}{\rule{0pt}{\xxtestlen}}}%
%    \global\setlength{\vshadelen}{0pt}%
%    \global\setlength{\pcalvspace}{0pt}}
   
   %%%% TESTING
   
   %% The xcpar environment
   %%  Note: overloaded use of \pcalvspace for testing.
   %%
%   \newlength{\xcparht}%
%   \newlength{\xcpardp}%
   
%   \newenvironment{xcpar}[6]{%
%  \addtocounter{pardepth}{-#1}%
%  \ifthenelse{\boolean{shading}}{\par\begin{lrbox}{\tempsbox}%
%                                 \begin{minipage}[t]{\linewidth}}{}%
%  \begin{list}{}{%
%     \edef\temp{\thegmargin}%
%     \ifthenelse{\equal{#3}{T}}%
%       {\settowidth{\leftmargin}{\hspace{\temp}\footnotesize #6\hspace{#5pt}}%
%        \addtolength{\leftmargin}{#4pt}}%
%       {\setlength{\leftmargin}{#4pt}%
%        \addtolength{\leftmargin}{#5pt}%
%        \addtolength{\leftmargin}{\temp}%
%        \setlength{\itemindent}{-#5pt}}%
%      \ifthenelse{\equal{#2}{T}}{\addtocounter{pardepth}{1}%
%                                 \setgmargin{\the\leftmargin}}{}%
%      \setlength{\labelwidth}{0pt}%
%      \setlength{\labelsep}{0pt}%
%      \setlength{\itemindent}{-\leftmargin}%
%      \setlength{\topsep}{0pt}%
%      \setlength{\parsep}{0pt}%
%      \setlength{\partopsep}{0pt}%
%      \setlength{\parskip}{0pt}%
%      \setlength{\itemsep}{0pt}%
%      \setlength{\itemindent}{#4pt}%
%      \addtolength{\itemindent}{-\leftmargin}}%
%   \ifthenelse{\equal{#3}{T}}%
%      {\item[\xtstrut\footnotesize \hspace{\temp}{#6}\hspace{#5pt}]%
%        }%
%      {\item[\xtstrut\hspace{\temp}]%
%         }%
%   \footnotesize}
% {\hspace{-\the\lastskip}\tstrut
% \end{list}%
%  \ifthenelse{\boolean{shading}}{\end{minipage}  
%                                 \end{lrbox}%
%                                 \setlength{\xcparht}{\ht\tempsbox}%
%                                 \setlength{\xcpardp}{\dp\tempsbox}%
%                                 \addtolength{\xcparht}{.15em}%
%                                 \addtolength{\xcpardp}{.2em}%
%                                 \addtolength{\xcparht}{\xcpardp}%
%                                 \hspace*{-\lcomindent}%
%                                 \makebox[0pt][l]{\raisebox{-\xcpardp}[0pt][0pt]{%
%                                      \shadebox{\rule{0pt}{\xcparht}\hspace*{\textwidth}}}}%
%                                 \hspace*{\lcomindent}\usebox{\tempsbox}%
%                                 \par}{}%
%  }
%  
% \newlength{\xmcomlen}
%\newenvironment{xmcom}[1]{%
%  \setcounter{pardepth}{0}%
%  \hspace{.65em}%
%  \begin{lrbox}{\alignbox}\sloppypar%
%      \setboolean{shading}{false}%
%      \setlength{\boxwidth}{#1pt}%
%      \addtolength{\boxwidth}{-.65em}%
%      \begin{minipage}[t]{\boxwidth}\footnotesize
%      \parskip=0pt\relax}%
%       {\end{minipage}\end{lrbox}%
%       \setlength{\xmcomlen}{\textwidth}%
%       \addtolength{\xmcomlen}{-\wd\alignbox}%
%       \settodepth{\alignwidth}{\usebox{\alignbox}}%
%       \global\setlength{\multicommentdepth}{\alignwidth}%
%       \setlength{\boxwidth}{\alignwidth}%
%       \global\addtolength{\alignwidth}{-\maxdepth}%
%       \addtolength{\boxwidth}{.1em}%
%       \raisebox{0pt}[0pt][0pt]{%
%        \ifthenelse{\boolean{shading}}%
%          {\hspace*{-\xmcomlen}\shadebox{\rule[-\boxwidth]{0pt}{0pt}%
%                                 \hspace*{\xmcomlen}\usebox{\alignbox}}}%
%          {\usebox{\alignbox}}}%
%       \vspace*{\alignwidth}\pagebreak[0]\vspace{-\alignwidth}\par}
% % a multi-line comment, whose first argument is its width in points.
%  
   
%%%%%%%%%%%%%%%%%%%%%%%%%%%%%%%%%%%%%%%%%%%%%%%%%%%%%%%%%%%%%%%%%%%%%%%%%%%%%%
% THE lcom ENVIRONMENT                                                       %
% ^^^^^^^^^^^^^^^^^^^^                                                       %
% A multi-line comment with no text to its left is typeset in an lcom        % 
% environment, whose argument is a number representing the indentation       % 
% of the left margin, in points.  All the text of the comment should be      % 
% inside cpar environments.                                                  % 
%%%%%%%%%%%%%%%%%%%%%%%%%%%%%%%%%%%%%%%%%%%%%%%%%%%%%%%%%%%%%%%%%%%%%%%%%%%%%%
\newenvironment{lcom}[1]{%
  \setlength{\lcomindent}{#1pt} % Added for PlusCal handling.
  \par\vspace{.2em}%
  \sloppypar
  \setcounter{pardepth}{0}%
  \footnotesize
  \begin{list}{}{%
    \setlength{\leftmargin}{#1pt}
    \setlength{\labelwidth}{0pt}%
    \setlength{\labelsep}{0pt}%
    \setlength{\itemindent}{0pt}%
    \setlength{\topsep}{0pt}%
    \setlength{\parsep}{0pt}%
    \setlength{\partopsep}{0pt}%
    \setlength{\parskip}{0pt}}
    \item[]}%
  {\end{list}\vspace{.3em}\setlength{\lcomindent}{0pt}%
 }


%%%%%%%%%%%%%%%%%%%%%%%%%%%%%%%%%%%%%%%%%%%%%%%%%%%%%%%%%%%%%%%%%%%%%%%%%%%%%
% THE mcom ENVIRONMENT AND \mutivspace COMMAND                              %
% ^^^^^^^^^^^^^^^^^^^^^^^^^^^^^^^^^^^^^^^^^^^^                              %
%                                                                           %
% A part of the spec containing a right-comment of the form                 %
%                                                                           %
%      xxxx (*************)                                                 %
%      yyyy (* ccccccccc *)                                                 %
%      ...  (* ccccccccc *)                                                 %
%           (* ccccccccc *)                                                 %
%           (* ccccccccc *)                                                 %
%           (*************)                                                 %
%                                                                           %
% is typeset by                                                             %
%                                                                           %
%     XXXX \begin{mcom}{d}                                                  %
%            CCCC ... CCC                                                   %
%          \end{mcom}                                                       %
%     YYYY ...                                                              %
%     \multivspace{n}                                                       %
%                                                                           %
% where the number d is the width in points of the comment, n is the        %
% number of xxxx, yyyy, ...  lines to the left of the comment.              %
% All the text of the comment should be typeset in cpar environments.       %
%                                                                           %
% This puts the comment into a single box (so no page breaks can occur      %
% within it).  The entire box is shaded iff the shading flag is true.       %
%%%%%%%%%%%%%%%%%%%%%%%%%%%%%%%%%%%%%%%%%%%%%%%%%%%%%%%%%%%%%%%%%%%%%%%%%%%%%
\newlength{\xmcomlen}%
\newenvironment{mcom}[1]{%
  \setcounter{pardepth}{0}%
  \hspace{.65em}%
  \begin{lrbox}{\alignbox}\sloppypar%
      \setboolean{shading}{false}%
      \setlength{\boxwidth}{#1pt}%
      \addtolength{\boxwidth}{-.65em}%
      \begin{minipage}[t]{\boxwidth}\footnotesize
      \parskip=0pt\relax}%
       {\end{minipage}\end{lrbox}%
       \setlength{\xmcomlen}{\textwidth}%       % For PlusCal shading
       \addtolength{\xmcomlen}{-\wd\alignbox}%  % For PlusCal shading
       \settodepth{\alignwidth}{\usebox{\alignbox}}%
       \global\setlength{\multicommentdepth}{\alignwidth}%
       \setlength{\boxwidth}{\alignwidth}%      % For PlusCal shading
       \global\addtolength{\alignwidth}{-\maxdepth}%
       \addtolength{\boxwidth}{.1em}%           % For PlusCal shading
      \raisebox{0pt}[0pt][0pt]{%
        \ifthenelse{\boolean{shading}}%
          {\ifpcalshading
             \hspace*{-\xmcomlen}%
             \shadebox{\rule[-\boxwidth]{0pt}{0pt}\hspace*{\xmcomlen}%
                          \usebox{\alignbox}}%
           \else
             \shadebox{\usebox{\alignbox}}
           \fi
          }%
          {\usebox{\alignbox}}}%
       \vspace*{\alignwidth}\pagebreak[0]\vspace{-\alignwidth}\par}
 % a multi-line comment, whose first argument is its width in points.


% \multispace{n} produces the vertical space indicated by "|"s in 
% this situation
%   
%     xxxx (*************)
%     xxxx (* ccccccccc *)
%      |   (* ccccccccc *)
%      |   (* ccccccccc *)
%      |   (* ccccccccc *)
%      |   (*************)
%
% where n is the number of "xxxx" lines.
\newcommand{\multivspace}[1]{\addtolength{\multicommentdepth}{-#1\baselineskip}%
 \addtolength{\multicommentdepth}{1.2em}%
 \ifthenelse{\lengthtest{\multicommentdepth > 0pt}}%
    {\par\vspace{\multicommentdepth}\par}{}}

%\newenvironment{hpar}[2]{%
%  \begin{list}{}{\setlength{\leftmargin}{#1pt}%
%                 \addtolength{\leftmargin}{#2pt}%
%                 \setlength{\itemindent}{-#2pt}%
%                 \setlength{\topsep}{0pt}%
%                 \setlength{\parsep}{0pt}%
%                 \setlength{\partopsep}{0pt}%
%                 \setlength{\parskip}{0pt}%
%                 \addtolength{\labelsep}{0pt}}%
%  \item[]\footnotesize}{\end{list}}
%    %%%%%%%%%%%%%%%%%%%%%%%%%%%%%%%%%%%%%%%%%%%%%%%%%%%%%%%%%%%%%%%%%%%%%%%%
%    % Typesets a sequence of paragraphs like this:                         %
%    %                                                                      %
%    %      left |<-- d1 --> XXXXXXXXXXXXXXXXXXXXXXXX                       %
%    %    margin |           <- d2 -> XXXXXXXXXXXXXXX                       %
%    %           |                    XXXXXXXXXXXXXXX                       %
%    %           |                                                          %
%    %           |                    XXXXXXXXXXXXXXX                       %
%    %           |                    XXXXXXXXXXXXXXX                       %
%    %                                                                      %
%    % where d1 = #1pt and d2 = #2pt, but with no vspace between            %
%    % paragraphs.                                                          %
%    %%%%%%%%%%%%%%%%%%%%%%%%%%%%%%%%%%%%%%%%%%%%%%%%%%%%%%%%%%%%%%%%%%%%%%%%

%%%%%%%%%%%%%%%%%%%%%%%%%%%%%%%%%%%%%%%%%%%%%%%%%%%%%%%%%%%%%%%%%%%%%%
% Commands for repeated characters that produce dashes.              %
%%%%%%%%%%%%%%%%%%%%%%%%%%%%%%%%%%%%%%%%%%%%%%%%%%%%%%%%%%%%%%%%%%%%%%
% \raisedDash{wd}{ht}{thk} makes a horizontal line wd characters wide, 
% raised a distance ht ex's above the baseline, with a thickness of 
% thk em's.
\newcommand{\raisedDash}[3]{\raisebox{#2ex}{\setlength{\alignwidth}{.5em}%
  \rule{#1\alignwidth}{#3em}}}

% The following commands take a single argument n and produce the
% output for n repeated characters, as follows
%   \cdash:    -
%   \tdash:    ~
%   \ceqdash:  =
%   \usdash:   _
\newcommand{\cdash}[1]{\raisedDash{#1}{.5}{.04}}
\newcommand{\usdash}[1]{\raisedDash{#1}{0}{.04}}
\newcommand{\ceqdash}[1]{\raisedDash{#1}{.5}{.08}}
\newcommand{\tdash}[1]{\raisedDash{#1}{1}{.08}}

\newlength{\spacewidth}
\setlength{\spacewidth}{.2em}
\newcommand{\e}[1]{\hspace{#1\spacewidth}}
%% \e{i} produces space corresponding to i input spaces.


%% Alignment-file Commands

\newlength{\alignboxwidth}
\newlength{\alignwidth}
\newsavebox{\alignbox}

% \al{i}{j}{txt} is used in the alignment file to put "%{i}{j}{wd}"
% in the log file, where wd is the width of the line up to that point,
% and txt is the following text.
\newcommand{\al}[3]{%
  \typeout{\%{#1}{#2}{\the\alignwidth}}%
  \cl{#3}}

%% \cl{txt} continues a specification line in the alignment file
%% with text txt.
\newcommand{\cl}[1]{%
  \savebox{\alignbox}{\mbox{$\mbox{}#1\mbox{}$}}%
  \settowidth{\alignboxwidth}{\usebox{\alignbox}}%
  \addtolength{\alignwidth}{\alignboxwidth}%
  \usebox{\alignbox}}

% \fl{txt} in the alignment file begins a specification line that
% starts with the text txt.
\newcommand{\fl}[1]{%
  \par
  \savebox{\alignbox}{\mbox{$\mbox{}#1\mbox{}$}}%
  \settowidth{\alignwidth}{\usebox{\alignbox}}%
  \usebox{\alignbox}}



  
%%%%%%%%%%%%%%%%%%%%%%%%%%%%%%%%%%%%%%%%%%%%%%%%%%%%%%%%%%%%%%%%%%%%%%%%%%%%%
% Ordinarily, TeX typesets letters in math mode in a special math italic    %
% font.  This makes it typeset "it" to look like the product of the         %
% variables i and t, rather than like the word "it".  The following         %
% commands tell TeX to use an ordinary italic font instead.                 %
%%%%%%%%%%%%%%%%%%%%%%%%%%%%%%%%%%%%%%%%%%%%%%%%%%%%%%%%%%%%%%%%%%%%%%%%%%%%%
\ifx\documentclass\undefined
\else
  \DeclareSymbolFont{tlaitalics}{\encodingdefault}{cmr}{m}{it}
  \let\itfam\symtlaitalics
\fi

\makeatletter
\newcommand{\tlx@c}{\c@tlx@ctr\advance\c@tlx@ctr\@ne}
\newcounter{tlx@ctr}
\c@tlx@ctr=\itfam \multiply\c@tlx@ctr"100\relax \advance\c@tlx@ctr "7061\relax
\mathcode`a=\tlx@c \mathcode`b=\tlx@c \mathcode`c=\tlx@c \mathcode`d=\tlx@c
\mathcode`e=\tlx@c \mathcode`f=\tlx@c \mathcode`g=\tlx@c \mathcode`h=\tlx@c
\mathcode`i=\tlx@c \mathcode`j=\tlx@c \mathcode`k=\tlx@c \mathcode`l=\tlx@c
\mathcode`m=\tlx@c \mathcode`n=\tlx@c \mathcode`o=\tlx@c \mathcode`p=\tlx@c
\mathcode`q=\tlx@c \mathcode`r=\tlx@c \mathcode`s=\tlx@c \mathcode`t=\tlx@c
\mathcode`u=\tlx@c \mathcode`v=\tlx@c \mathcode`w=\tlx@c \mathcode`x=\tlx@c
\mathcode`y=\tlx@c \mathcode`z=\tlx@c
\c@tlx@ctr=\itfam \multiply\c@tlx@ctr"100\relax \advance\c@tlx@ctr "7041\relax
\mathcode`A=\tlx@c \mathcode`B=\tlx@c \mathcode`C=\tlx@c \mathcode`D=\tlx@c
\mathcode`E=\tlx@c \mathcode`F=\tlx@c \mathcode`G=\tlx@c \mathcode`H=\tlx@c
\mathcode`I=\tlx@c \mathcode`J=\tlx@c \mathcode`K=\tlx@c \mathcode`L=\tlx@c
\mathcode`M=\tlx@c \mathcode`N=\tlx@c \mathcode`O=\tlx@c \mathcode`P=\tlx@c
\mathcode`Q=\tlx@c \mathcode`R=\tlx@c \mathcode`S=\tlx@c \mathcode`T=\tlx@c
\mathcode`U=\tlx@c \mathcode`V=\tlx@c \mathcode`W=\tlx@c \mathcode`X=\tlx@c
\mathcode`Y=\tlx@c \mathcode`Z=\tlx@c
\makeatother

%%%%%%%%%%%%%%%%%%%%%%%%%%%%%%%%%%%%%%%%%%%%%%%%%%%%%%%%%%
%                THE describe ENVIRONMENT                %
%%%%%%%%%%%%%%%%%%%%%%%%%%%%%%%%%%%%%%%%%%%%%%%%%%%%%%%%%%
%
%
% It is like the description environment except it takes an argument
% ARG that should be the text of the widest label.  It adjusts the
% indentation so each item with label LABEL produces
%%      LABEL             blah blah blah
%%      <- width of ARG ->blah blah blah
%%                        blah blah blah
\newenvironment{describe}[1]%
   {\begin{list}{}{\settowidth{\labelwidth}{#1}%
            \setlength{\labelsep}{.5em}%
            \setlength{\leftmargin}{\labelwidth}% 
            \addtolength{\leftmargin}{\labelsep}%
            \addtolength{\leftmargin}{\parindent}%
            \def\makelabel##1{\rm ##1\hfill}}%
            \setlength{\topsep}{0pt}}%% 
                % Sets \topsep to 0 to reduce vertical space above
                % and below embedded displayed equations
   {\end{list}}

%   For tlatex.TeX
\usepackage{verbatim}
\makeatletter
\def\tla{\let\%\relax%
         \@bsphack
         \typeout{\%{\the\linewidth}}%
             \let\do\@makeother\dospecials\catcode`\^^M\active
             \let\verbatim@startline\relax
             \let\verbatim@addtoline\@gobble
             \let\verbatim@processline\relax
             \let\verbatim@finish\relax
             \verbatim@}
\let\endtla=\@esphack

\let\pcal=\tla
\let\endpcal=\endtla
\let\ppcal=\tla
\let\endppcal=\endtla

% The tlatex environment is used by TLATeX.TeX to typeset TLA+.
% TLATeX.TLA starts its files by writing a \tlatex command.  This
% command/environment sets \parindent to 0 and defines \% to its
% standard definition because the writing of the log files is messed up
% if \% is defined to be something else.  It also executes
% \@computerule to determine the dimensions for the TLA horizonatl
% bars.
\newenvironment{tlatex}{\@computerule%
                        \setlength{\parindent}{0pt}%
                       \makeatletter\chardef\%=`\%}{}


% The notla environment produces no output.  You can turn a 
% tla environment to a notla environment to prevent tlatex.TeX from
% re-formatting the environment.

\def\notla{\let\%\relax%
         \@bsphack
             \let\do\@makeother\dospecials\catcode`\^^M\active
             \let\verbatim@startline\relax
             \let\verbatim@addtoline\@gobble
             \let\verbatim@processline\relax
             \let\verbatim@finish\relax
             \verbatim@}
\let\endnotla=\@esphack

\let\nopcal=\notla
\let\endnopcal=\endnotla
\let\noppcal=\notla
\let\endnoppcal=\endnotla

%%%%%%%%%%%%%%%%%%%%%%%% end of tlatex.sty file %%%%%%%%%%%%%%%%%%%%%%% 
% last modified on Fri  3 August 2012 at 14:23:49 PST by lamport

\begin{document}
\tlatex
\setboolean{shading}{true}
 \@x{\makebox[0pt][r]{\scriptsize 1\hspace{1em}}}\moduleLeftDash\@xx{
 {\MODULE} Paxos}\moduleRightDash\@xx{}%
\begin{lcom}{0}%
\begin{cpar}{0}{F}{F}{0}{0}{}%
Specification and Verification of Basic \ensuremath{Paxos}.
\end{cpar}%
\vshade{5.0}%
\begin{cpar}{0}{F}{F}{0}{0}{}%
See http:\ensuremath{\.{\slsl}research.microsoft.com}/en-us/um/people/lamport/pubs/\ensuremath{pubs.html\.{\neq}}paxos-simple
\end{cpar}%
\end{lcom}%
 \@x{\makebox[0pt][r]{\scriptsize 7\hspace{1em}} {\EXTENDS} Integers ,\, TLAPS
 ,\, TLC}%
\@x{\makebox[0pt][r]{\scriptsize 8\hspace{1em}}}\midbar\@xx{}%
 \@x{\makebox[0pt][r]{\scriptsize 9\hspace{1em}} {\CONSTANTS} Acceptors ,\,
 Values ,\, Quorums}%
\@pvspace{8.0pt}%
 \@x{\makebox[0pt][r]{\scriptsize 11\hspace{1em}} {\ASSUME} QuorumAssumption
 \.{\defeq}}%
 \@x{\makebox[0pt][r]{\scriptsize 12\hspace{1em}}\@s{50.54} \.{\land} Quorums
 \.{\subseteq} {\SUBSET} Acceptors}%
 \@x{\makebox[0pt][r]{\scriptsize 13\hspace{1em}}\@s{50.54} \.{\land} \A\, Q1
 ,\, Q2 \.{\in} Quorums \.{:} Q1 \.{\cap} Q2 \.{\neq} \{ \}}%
\@pvspace{8.0pt}%
 \@x{\makebox[0pt][r]{\scriptsize 15\hspace{1em}} {\LEMMA} QuorumNonEmpty
 \.{\defeq} \A\, Q \.{\in} Quorums \.{:} Q \.{\neq} \{ \}}%
\@x{\makebox[0pt][r]{\scriptsize 16\hspace{1em}} {\BY} QuorumAssumption}%
\@pvspace{8.0pt}%
\@x{\makebox[0pt][r]{\scriptsize 18\hspace{1em}} Ballots \.{\defeq} Nat}%
\@pvspace{8.0pt}%
 \@x{\makebox[0pt][r]{\scriptsize 20\hspace{1em}} None \.{\defeq} {\CHOOSE} v
 \.{:} v \.{\notin} Values}%
\@pvspace{8.0pt}%
 \@x{\makebox[0pt][r]{\scriptsize 22\hspace{1em}} {\LEMMA} NoneNotAValue
 \.{\defeq} None \.{\notin} Values}%
 \@x{\makebox[0pt][r]{\scriptsize 23\hspace{1em}} {\BY}
 NoSetContainsEverything {\DEF} None}%
\@pvspace{8.0pt}%
 \@x{\makebox[0pt][r]{\scriptsize 25\hspace{1em}} Messages \.{\defeq}\@s{20.5}
 [ type \.{:} \{\@w{1a} \} ,\, bal\@s{0.36} \.{:} Ballots ]}%
 \@x{\makebox[0pt][r]{\scriptsize 26\hspace{1em}}\@s{53.26} \.{\cup}\@s{15.97}
 [ type \.{:} \{\@w{1b} \} ,\, bal \.{:} Ballots ,\, maxVBal \.{:} Ballots
 \.{\cup} \{ \.{-} 1 \} ,\,}%
 \@x{\makebox[0pt][r]{\scriptsize 27\hspace{1em}}\@s{91.33} maxVal \.{:}
 Values \.{\cup} \{ None \} ,\, acc \.{:} Acceptors ]}%
 \@x{\makebox[0pt][r]{\scriptsize 28\hspace{1em}}\@s{53.26} \.{\cup}\@s{15.97}
 [ type \.{:} \{\@w{2a} \} ,\, bal\@s{0.36} \.{:} Ballots ,\, val \.{:}
 Values ]}%
 \@x{\makebox[0pt][r]{\scriptsize 29\hspace{1em}}\@s{53.26} \.{\cup}\@s{15.97}
 [ type \.{:} \{\@w{2b} \} ,\, bal \.{:} Ballots ,\, val \.{:} Values ,\, acc
 \.{:} Acceptors ]}%
\@x{\makebox[0pt][r]{\scriptsize 30\hspace{1em}}}\midbar\@xx{}%
 \@x{\makebox[0pt][r]{\scriptsize 31\hspace{1em}} {\VARIABLES} msgs
 ,\,\@s{19.32}}%
\@y{\@s{0}%
 the set of messages that have been sent.
}%
\@xx{}%
 \@x{\makebox[0pt][r]{\scriptsize 32\hspace{1em}}\@s{51.42} maxBal
 ,\,\@s{7.43}}%
\@y{\@s{0}%
 \ensuremath{maxBal[a]}: the highest-number ballot acceptor a has
 participated in.
}%
\@xx{}%
\@x{\makebox[0pt][r]{\scriptsize 33\hspace{1em}}\@s{51.42} maxVBal ,\,}%
\@y{\@s{0}%
 \ensuremath{maxVBal[a]}: the highest ballot in which a has voted;
}%
\@xx{}%
\@x{\makebox[0pt][r]{\scriptsize 34\hspace{1em}}\@s{51.42} maxVal\@s{13.91}}%
\@y{\@s{0}%
 \ensuremath{maxVal[a]}: the value it voted for in that ballot.
}%
\@xx{}%
\@pvspace{8.0pt}%
 \@x{\makebox[0pt][r]{\scriptsize 36\hspace{1em}} vars \.{\defeq} {\langle}
 msgs ,\, maxBal ,\, maxVBal ,\, maxVal {\rangle}}%
\@pvspace{8.0pt}%
 \@x{\makebox[0pt][r]{\scriptsize 38\hspace{1em}} TypeOK \.{\defeq} \.{\land}
 msgs \.{\in} {\SUBSET} Messages}%
 \@x{\makebox[0pt][r]{\scriptsize 39\hspace{1em}}\@s{56.14} \.{\land} maxVBal
 \.{\in} [ Acceptors \.{\rightarrow} Ballots \.{\cup} \{ \.{-} 1 \} ]}%
 \@x{\makebox[0pt][r]{\scriptsize 40\hspace{1em}}\@s{56.14} \.{\land} maxBal
 \.{\in}\@s{4.1} [ Acceptors\@s{3.33} \.{\rightarrow} Ballots \.{\cup} \{
 \.{-} 1 \} ]}%
 \@x{\makebox[0pt][r]{\scriptsize 41\hspace{1em}}\@s{56.14} \.{\land}
 maxVal\@s{0.37} \.{\in}\@s{4.09} [ Acceptors\@s{3.33} \.{\rightarrow} Values
 \.{\cup} \{ None \} ]}%
 \@x{\makebox[0pt][r]{\scriptsize 42\hspace{1em}}\@s{56.14} \.{\land} \A\, a
 \.{\in} Acceptors \.{:} maxBal [ a ] \.{\geq} maxVBal [ a ]}%
\@pvspace{8.0pt}%
 \@x{\makebox[0pt][r]{\scriptsize 44\hspace{1em}} Send ( m ) \.{\defeq} msgs
 \.{'} \.{=} msgs \.{\cup} \{ m \}}%
\@x{\makebox[0pt][r]{\scriptsize 45\hspace{1em}}}\midbar\@xx{}%
 \@x{\makebox[0pt][r]{\scriptsize 46\hspace{1em}} Init \.{\defeq} \.{\land}
 msgs \.{=} \{ \}}%
 \@x{\makebox[0pt][r]{\scriptsize 47\hspace{1em}}\@s{35.70} \.{\land} maxVBal
 \.{=} [ a \.{\in} Acceptors \.{\mapsto} \.{-} 1 ]}%
 \@x{\makebox[0pt][r]{\scriptsize 48\hspace{1em}}\@s{35.70} \.{\land}
 maxBal\@s{7.43} \.{=} [ a \.{\in} Acceptors \.{\mapsto} \.{-} 1 ]}%
 \@x{\makebox[0pt][r]{\scriptsize 49\hspace{1em}}\@s{35.70} \.{\land}
 maxVal\@s{7.80} \.{=} [ a \.{\in} Acceptors \.{\mapsto} None ]}%
\@pvspace{8.0pt}%
 \@x{\makebox[0pt][r]{\scriptsize 51\hspace{1em}} Phase1a ( b ) \.{\defeq}
 \.{\land} {\lnot} \E\, m\@s{6.73} \.{\in} msgs \.{:} ( m . type \.{=}\@w{1a}
 ) \.{\land} ( m . bal \.{=} b )}%
 \@x{\makebox[0pt][r]{\scriptsize 52\hspace{1em}}\@s{69.22} \.{\land} Send ( [
 type \.{\mapsto}\@w{1a} ,\, bal \.{\mapsto} b ] )}%
 \@x{\makebox[0pt][r]{\scriptsize 53\hspace{1em}}\@s{69.22} \.{\land}
 {\UNCHANGED} {\langle} maxVBal ,\, maxBal ,\, maxVal {\rangle}}%
\@pvspace{8.0pt}%
\@x{\makebox[0pt][r]{\scriptsize 55\hspace{1em}} Phase1b ( a ) \.{\defeq}}%
 \@x{\makebox[0pt][r]{\scriptsize 56\hspace{1em}}\@s{8.2} \E\, m \.{\in} msgs
 \.{:}}%
 \@x{\makebox[0pt][r]{\scriptsize 57\hspace{1em}}\@s{15.42} \.{\land} m . type
 \.{=}\@w{1a}}%
 \@x{\makebox[0pt][r]{\scriptsize 58\hspace{1em}}\@s{15.42} \.{\land} m . bal
 \.{>} maxBal [ a ]}%
 \@x{\makebox[0pt][r]{\scriptsize 59\hspace{1em}}\@s{15.42} \.{\land} Send ( [
 type \.{\mapsto}\@w{1b} ,\, bal \.{\mapsto} m . bal ,\, maxVBal \.{\mapsto}
 maxVBal [ a ] ,\,}%
 \@x{\makebox[0pt][r]{\scriptsize 60\hspace{1em}}\@s{59.28} maxVal \.{\mapsto}
 maxVal [ a ] ,\, acc \.{\mapsto} a ] )}%
 \@x{\makebox[0pt][r]{\scriptsize 61\hspace{1em}}\@s{15.42} \.{\land} maxBal
 \.{'} \.{=} [ maxBal {\EXCEPT} {\bang} [ a ] \.{=} m . bal ]}%
 \@x{\makebox[0pt][r]{\scriptsize 62\hspace{1em}}\@s{15.42} \.{\land}
 {\UNCHANGED} {\langle} maxVBal ,\, maxVal {\rangle}}%
\@pvspace{8.0pt}%
\@x{\makebox[0pt][r]{\scriptsize 64\hspace{1em}} Phase2a ( b ) \.{\defeq}}%
 \@x{\makebox[0pt][r]{\scriptsize 65\hspace{1em}}\@s{8.2} \.{\land} {\lnot}
 \E\, m \.{\in} msgs \.{:} ( m . type \.{=}\@w{2a} ) \.{\land} ( m . bal
 \.{=} b )}%
 \@x{\makebox[0pt][r]{\scriptsize 66\hspace{1em}}\@s{8.2} \.{\land} \E\, v
 \.{\in} Values\@s{2.50} \.{:}}%
 \@x{\makebox[0pt][r]{\scriptsize 67\hspace{1em}}\@s{27.51} \.{\land} \E\, Q
 \.{\in} Quorums \.{:}}%
 \@x{\makebox[0pt][r]{\scriptsize 68\hspace{1em}}\@s{46.82} \E\, S \.{\in}
 {\SUBSET} \{ m \.{\in} msgs \.{:} ( m . type \.{=}\@w{1b} ) \.{\land} ( m .
 bal \.{=} b ) \} \.{:}}%
 \@x{\makebox[0pt][r]{\scriptsize 69\hspace{1em}}\@s{54.04} \.{\land} \A\, a
 \.{\in} Q \.{:} \E\, m \.{\in} S \.{:} m . acc\@s{22.41} \.{=} a}%
 \@x{\makebox[0pt][r]{\scriptsize 70\hspace{1em}}\@s{54.04} \.{\land} \.{\lor}
 \A\, m \.{\in} S \.{:} m . maxVBal \.{=} \.{-} 1}%
 \@x{\makebox[0pt][r]{\scriptsize 71\hspace{1em}}\@s{65.15} \.{\lor} \E\,
 c\@s{3.77} \.{\in} 0 \.{\dotdot} ( b \.{-} 1 ) \.{:}}%
 \@x{\makebox[0pt][r]{\scriptsize 72\hspace{1em}}\@s{83.48} \.{\land} \A\, m
 \.{\in} S \.{:} m . maxVBal \.{\leq} c}%
 \@x{\makebox[0pt][r]{\scriptsize 73\hspace{1em}}\@s{83.48} \.{\land} \E\, m
 \.{\in} S \.{:} \.{\land} m . maxVBal \.{=} c}%
 \@x{\makebox[0pt][r]{\scriptsize 74\hspace{1em}}\@s{139.25} \.{\land} m .
 maxVal \.{=} v}%
 \@x{\makebox[0pt][r]{\scriptsize 75\hspace{1em}}\@s{27.51} \.{\land} Send ( [
 type \.{\mapsto}\@w{2a} ,\, bal \.{\mapsto} b ,\, val \.{\mapsto} v ] )}%
 \@x{\makebox[0pt][r]{\scriptsize 76\hspace{1em}}\@s{8.2} \.{\land}
 {\UNCHANGED} {\langle} maxBal ,\, maxVBal ,\, maxVal {\rangle}}%
\@pvspace{8.0pt}%
\@x{\makebox[0pt][r]{\scriptsize 78\hspace{1em}} Phase2b ( a ) \.{\defeq}}%
 \@x{\makebox[0pt][r]{\scriptsize 79\hspace{1em}}\@s{8.2} \E\, m \.{\in} msgs
 \.{:}}%
 \@x{\makebox[0pt][r]{\scriptsize 80\hspace{1em}}\@s{16.4} \.{\land} m . type
 \.{=}\@w{2a}}%
 \@x{\makebox[0pt][r]{\scriptsize 81\hspace{1em}}\@s{16.4} \.{\land} m . bal
 \.{\geq} maxBal [ a ]}%
 \@x{\makebox[0pt][r]{\scriptsize 82\hspace{1em}}\@s{16.4} \.{\land} Send ( [
 type \.{\mapsto}\@w{2b} ,\, bal \.{\mapsto} m . bal ,\, val \.{\mapsto} m .
 val ,\, acc \.{\mapsto} a ] )}%
 \@x{\makebox[0pt][r]{\scriptsize 83\hspace{1em}}\@s{16.4} \.{\land} maxVBal
 \.{'} \.{=} [ maxVBal {\EXCEPT} {\bang} [ a ] \.{=} m . bal ]}%
 \@x{\makebox[0pt][r]{\scriptsize 84\hspace{1em}}\@s{16.4} \.{\land} maxBal
 \.{'} \.{=} [ maxBal {\EXCEPT} {\bang} [ a ] \.{=} m . bal ]}%
 \@x{\makebox[0pt][r]{\scriptsize 85\hspace{1em}}\@s{16.4} \.{\land} maxVal
 \.{'}\@s{0.37} \.{=} [ maxVal {\EXCEPT} {\bang} [ a ]\@s{0.37} \.{=} m . val
 ]}%
\@x{\makebox[0pt][r]{\scriptsize 86\hspace{1em}}}\midbar\@xx{}%
 \@x{\makebox[0pt][r]{\scriptsize 87\hspace{1em}} Next \.{\defeq} \.{\lor}
 \E\, b\@s{0.64} \.{\in} Ballots \.{:} Phase1a ( b ) \.{\lor} Phase2a ( b )}%
 \@x{\makebox[0pt][r]{\scriptsize 88\hspace{1em}}\@s{39.83} \.{\lor} \E\, a
 \.{\in} Acceptors \.{:} Phase1b ( a ) \.{\lor} Phase2b ( a )}%
\@pvspace{8.0pt}%
 \@x{\makebox[0pt][r]{\scriptsize 90\hspace{1em}} Spec\@s{1.46} \.{\defeq}
 Init \.{\land} {\Box} [ Next ]_{ vars}}%
\@x{\makebox[0pt][r]{\scriptsize 91\hspace{1em}}}\midbar\@xx{}%
 \@x{\makebox[0pt][r]{\scriptsize 92\hspace{1em}} VotedForIn ( a ,\, v ,\, b )
 \.{\defeq} \E\, m \.{\in} msgs \.{:} \.{\land} m . type \.{=}\@w{2b}}%
 \@x{\makebox[0pt][r]{\scriptsize 93\hspace{1em}}\@s{164.91} \.{\land} m .
 val\@s{5.16} \.{=} v}%
 \@x{\makebox[0pt][r]{\scriptsize 94\hspace{1em}}\@s{164.91} \.{\land} m .
 bal\@s{5.67} \.{=} b}%
 \@x{\makebox[0pt][r]{\scriptsize 95\hspace{1em}}\@s{164.91} \.{\land} m .
 acc\@s{3.76} \.{=} a}%
\@pvspace{8.0pt}%
 \@x{\makebox[0pt][r]{\scriptsize 97\hspace{1em}} ChosenIn ( v ,\, b )
 \.{\defeq} \E\, Q \.{\in} Quorums \.{:}}%
 \@x{\makebox[0pt][r]{\scriptsize 98\hspace{1em}}\@s{92.84} \A\, a \.{\in} Q
 \.{:} VotedForIn ( a ,\, v ,\, b )}%
\@pvspace{8.0pt}%
 \@x{\makebox[0pt][r]{\scriptsize 100\hspace{1em}} Chosen ( v ) \.{\defeq}
 \E\, b \.{\in} Ballots \.{:} ChosenIn ( v ,\, b )}%
\@pvspace{8.0pt}%
 \@x{\makebox[0pt][r]{\scriptsize 102\hspace{1em}} Consistency \.{\defeq} \A\,
 v1 ,\, v2 \.{\in} Values \.{:} Chosen ( v1 ) \.{\land} Chosen ( v2 )
 \.{\implies} ( v1 \.{=} v2 )}%
\@x{\makebox[0pt][r]{\scriptsize 103\hspace{1em}}}\midbar\@xx{}%
 \@x{\makebox[0pt][r]{\scriptsize 104\hspace{1em}} WontVoteIn ( a ,\, b )
 \.{\defeq} \.{\land} \A\, v \.{\in} Values \.{:} {\lnot} VotedForIn ( a ,\,
 v ,\, b )}%
 \@x{\makebox[0pt][r]{\scriptsize 105\hspace{1em}}\@s{97.87} \.{\land} maxBal
 [ a ] \.{>} b}%
\@pvspace{8.0pt}%
 \@x{\makebox[0pt][r]{\scriptsize 107\hspace{1em}} SafeAt ( v ,\, b )
 \.{\defeq}}%
 \@x{\makebox[0pt][r]{\scriptsize 108\hspace{1em}}\@s{8.2} \A\, c \.{\in} 0
 \.{\dotdot} ( b \.{-} 1 ) \.{:}}%
 \@x{\makebox[0pt][r]{\scriptsize 109\hspace{1em}}\@s{16.4} \E\, Q \.{\in}
 Quorums \.{:}}%
 \@x{\makebox[0pt][r]{\scriptsize 110\hspace{1em}}\@s{24.59} \A\, a \.{\in} Q
 \.{:} VotedForIn ( a ,\, v ,\, c ) \.{\lor} WontVoteIn ( a ,\, c )}%
\@x{\makebox[0pt][r]{\scriptsize 111\hspace{1em}}}\midbar\@xx{}%
\@x{\makebox[0pt][r]{\scriptsize 112\hspace{1em}} MsgInv \.{\defeq}}%
 \@x{\makebox[0pt][r]{\scriptsize 113\hspace{1em}}\@s{8.2} \A\, m \.{\in} msgs
 \.{:}}%
 \@x{\makebox[0pt][r]{\scriptsize 114\hspace{1em}}\@s{16.4} \.{\land} ( m .
 type \.{=}\@w{1b} ) \.{\implies} \.{\land} m . bal \.{\leq} maxBal [ m . acc
 ]}%
 \@x{\makebox[0pt][r]{\scriptsize 115\hspace{1em}}\@s{114.19} \.{\land}
 \.{\lor} \.{\land} m . maxVal \.{\in} Values}%
 \@x{\makebox[0pt][r]{\scriptsize 116\hspace{1em}}\@s{136.41} \.{\land} m .
 maxVBal \.{\in} Ballots}%
\@x{\makebox[0pt][r]{\scriptsize 117\hspace{1em}}\@s{136.41}}%
\@y{\@s{0}%
 conjunct strengthened 2014/04/02 sm
}%
\@xx{}%
 \@x{\makebox[0pt][r]{\scriptsize 118\hspace{1em}}\@s{136.41} \.{\land}
 VotedForIn ( m . acc ,\, m . maxVal ,\, m . maxVBal )}%
\@x{\makebox[0pt][r]{\scriptsize 119\hspace{1em}}}%
\@y{\@s{77.5}%
 \ensuremath{\.{\land} SafeAt(m.maxVal,\, m.maxVBal)
}}%
\@xx{}%
 \@x{\makebox[0pt][r]{\scriptsize 120\hspace{1em}}\@s{118.9} \.{\lor}
 \.{\land} m . maxVal \.{=} None}%
 \@x{\makebox[0pt][r]{\scriptsize 121\hspace{1em}}\@s{130.01} \.{\land} m .
 maxVBal \.{=} \.{-} 1}%
\@x{\makebox[0pt][r]{\scriptsize 122\hspace{1em}}\@s{106.6}}%
\@y{%
 * conjunct added 2014/03/29 sm
}%
\@xx{}%
 \@x{\makebox[0pt][r]{\scriptsize 123\hspace{1em}}\@s{106.6} \.{\land} \A\, c
 \.{\in} ( m . maxVBal \.{+} 1 ) \.{\dotdot} ( m . bal \.{-} 1 ) \.{:}}%
 \@x{\makebox[0pt][r]{\scriptsize 124\hspace{1em}}\@s{124.93} {\lnot} \E\, v
 \.{\in} Values \.{:} VotedForIn ( m . acc ,\, v ,\, c )}%
 \@x{\makebox[0pt][r]{\scriptsize 125\hspace{1em}}\@s{16.4} \.{\land} ( m .
 type \.{=}\@w{2a} ) \.{\implies}}%
 \@x{\makebox[0pt][r]{\scriptsize 126\hspace{1em}}\@s{40.34} \.{\land} SafeAt
 ( m . val ,\, m . bal )}%
 \@x{\makebox[0pt][r]{\scriptsize 127\hspace{1em}}\@s{40.34} \.{\land} \A\, ma
 \.{\in} msgs \.{:} ( ma . type \.{=}\@w{2a} ) \.{\land} ( ma . bal \.{=} m .
 bal )}%
 \@x{\makebox[0pt][r]{\scriptsize 128\hspace{1em}}\@s{134.12} \.{\implies} (
 ma \.{=} m )}%
 \@x{\makebox[0pt][r]{\scriptsize 129\hspace{1em}}\@s{16.4} \.{\land} ( m .
 type \.{=}\@w{2b} ) \.{\implies}}%
 \@x{\makebox[0pt][r]{\scriptsize 130\hspace{1em}}\@s{40.34} \.{\land} \E\,
 ma\@s{3.23} \.{\in} msgs \.{:} \.{\land} ma . type \.{=}\@w{2a}}%
 \@x{\makebox[0pt][r]{\scriptsize 131\hspace{1em}}\@s{119.40} \.{\land} ma .
 bal\@s{5.34} \.{=} m . bal}%
 \@x{\makebox[0pt][r]{\scriptsize 132\hspace{1em}}\@s{119.40} \.{\land} ma .
 val\@s{4.82} \.{=} m . val}%
 \@x{\makebox[0pt][r]{\scriptsize 133\hspace{1em}}\@s{40.34} \.{\land} m . bal
 \.{\leq} maxVBal [ m . acc ]}%
\@x{\makebox[0pt][r]{\scriptsize 134\hspace{1em}}}\midbar\@xx{}%
 \@x{\makebox[0pt][r]{\scriptsize 135\hspace{1em}} {\LEMMA} VotedInv
 \.{\defeq}}%
 \@x{\makebox[0pt][r]{\scriptsize 136\hspace{1em}}\@s{43.75} MsgInv \.{\land}
 TypeOK \.{\implies}}%
 \@x{\makebox[0pt][r]{\scriptsize 137\hspace{1em}}\@s{60.15} \A\, a \.{\in}
 Acceptors ,\, v \.{\in} Values ,\, b \.{\in} Ballots \.{:}}%
 \@x{\makebox[0pt][r]{\scriptsize 138\hspace{1em}}\@s{71.47} VotedForIn ( a
 ,\, v ,\, b ) \.{\implies} SafeAt ( v ,\, b ) \.{\land} b \.{\leq} maxVBal [
 a ]}%
 \@x{\makebox[0pt][r]{\scriptsize 139\hspace{1em}} {\BY} {\DEF} VotedForIn ,\,
 MsgInv ,\, Messages ,\, TypeOK}%
\@pvspace{8.0pt}%
 \@x{\makebox[0pt][r]{\scriptsize 141\hspace{1em}} {\LEMMA} VotedOnce
 \.{\defeq}}%
 \@x{\makebox[0pt][r]{\scriptsize 142\hspace{1em}}\@s{43.75} MsgInv
 \.{\implies}\@s{4.1} \A\, a1 ,\, a2 \.{\in} Acceptors ,\, b \.{\in} Ballots
 ,\, v1 ,\, v2 \.{\in} Values \.{:}}%
 \@x{\makebox[0pt][r]{\scriptsize 143\hspace{1em}}\@s{107.54} VotedForIn ( a1
 ,\, v1 ,\, b ) \.{\land} VotedForIn ( a2 ,\, v2 ,\, b ) \.{\implies} ( v1
 \.{=} v2 )}%
 \@x{\makebox[0pt][r]{\scriptsize 144\hspace{1em}} {\BY} {\DEF} MsgInv ,\,
 VotedForIn}%
\@pvspace{8.0pt}%
\@x{\makebox[0pt][r]{\scriptsize 146\hspace{1em}} AccInv \.{\defeq}}%
 \@x{\makebox[0pt][r]{\scriptsize 147\hspace{1em}}\@s{8.2} \A\, a \.{\in}
 Acceptors \.{:}}%
 \@x{\makebox[0pt][r]{\scriptsize 148\hspace{1em}}\@s{16.4} \.{\land} ( maxVal
 [ a ]\@s{3.91} \.{=} None ) \.{\equiv} ( maxVBal [ a ] \.{=} \.{-} 1 )}%
 \@x{\makebox[0pt][r]{\scriptsize 149\hspace{1em}}\@s{16.4} \.{\land} maxVBal
 [ a ] \.{\leq} maxBal [ a ]}%
\@x{\makebox[0pt][r]{\scriptsize 150\hspace{1em}}\@s{16.4}}%
\@y{\@s{0}%
 conjunct strengthened corresponding to \ensuremath{MsgInv} 2014/04/02 sm
}%
\@xx{}%
 \@x{\makebox[0pt][r]{\scriptsize 151\hspace{1em}}\@s{16.4} \.{\land} (
 maxVBal [ a ] \.{\geq} 0 ) \.{\implies} VotedForIn ( a ,\, maxVal [ a ] ,\,
 maxVBal [ a ] )\@s{4.1}}%
\@y{\@s{0}%
 \ensuremath{SafeAt(maxVal[a],\, maxVBal[a])
}}%
\@xx{}%
\@x{\makebox[0pt][r]{\scriptsize 152\hspace{1em}}\@s{16.4}}%
\@y{\@s{0}%
 conjunct added corresponding to \ensuremath{MsgInv} 2014/03/29 sm
}%
\@xx{}%
 \@x{\makebox[0pt][r]{\scriptsize 153\hspace{1em}}\@s{16.4} \.{\land} \A\, c
 \.{\in} Ballots \.{:} c \.{>} maxVBal [ a ] \.{\implies} {\lnot} \E\, v
 \.{\in} Values \.{:} VotedForIn ( a ,\, v ,\, c )}%
\@x{\makebox[0pt][r]{\scriptsize 154\hspace{1em}}}\midbar\@xx{}%
 \@x{\makebox[0pt][r]{\scriptsize 155\hspace{1em}} Inv \.{\defeq} TypeOK
 \.{\land} MsgInv \.{\land} AccInv}%
\@x{\makebox[0pt][r]{\scriptsize 156\hspace{1em}}}\midbar\@xx{}%
\begin{lcom}{0}%
\begin{cpar}{0}{F}{F}{0}{0}{}%
The following lemma shows that (the invariant implies that) the
 predicate \ensuremath{SafeAt(v,\, b)} is stable, meaning that once it
 becomes true, it
 remains true throughout the rest of the excecution.
\end{cpar}%
\end{lcom}%
 \@x{\makebox[0pt][r]{\scriptsize 162\hspace{1em}} {\LEMMA} SafeAtStable
 \.{\defeq} Inv \.{\land} Next \.{\land} TypeOK \.{'} \.{\implies}}%
 \@x{\makebox[0pt][r]{\scriptsize 163\hspace{1em}}\@s{125.06} \A\, v \.{\in}
 Values ,\, b \.{\in} Ballots \.{:}}%
 \@x{\makebox[0pt][r]{\scriptsize 164\hspace{1em}}\@s{150.25} SafeAt ( v ,\, b
 ) \.{\implies} SafeAt ( v ,\, b ) \.{'}}%
 \@x{\makebox[0pt][r]{\scriptsize 165\hspace{1em}}\@pfstepnum{1}{}\ 
 {\SUFFICES} {\ASSUME} Inv ,\, Next ,\, TypeOK \.{'} ,\,}%
 \@x{\makebox[0pt][r]{\scriptsize 166\hspace{1em}}\@s{98.64} {\NEW} v \.{\in}
 Values ,\, {\NEW} b \.{\in} Ballots ,\, SafeAt ( v ,\, b )}%
 \@x{\makebox[0pt][r]{\scriptsize 167\hspace{1em}}\@s{60.39} {\PROVE}\@s{4.1}
 SafeAt ( v ,\, b ) \.{'}}%
\@x{\makebox[0pt][r]{\scriptsize 168\hspace{1em}}\@s{8.2} {\OBVIOUS}}%
 \@x{\makebox[0pt][r]{\scriptsize 169\hspace{1em}}\@pfstepnum{1}{}\  {\USE}
 {\DEF} Send ,\, Inv ,\, Ballots}%
 \@x{\makebox[0pt][r]{\scriptsize 170\hspace{1em}}\@pfstepnum{1}{}\  {\USE}
 {\TRUE} \.{\land} {\TRUE}}%
 \@x{\makebox[0pt][r]{\scriptsize 171\hspace{1em}}\@pfstepnum{1}{1.}\ 
 {\ASSUME} {\NEW} bb \.{\in} Ballots ,\, Phase1a ( bb )}%
 \@x{\makebox[0pt][r]{\scriptsize 172\hspace{1em}}\@s{23.88} {\PROVE}\@s{4.1}
 SafeAt ( v ,\, b ) \.{'}}%
 \@x{\makebox[0pt][r]{\scriptsize 173\hspace{1em}}\@s{8.2}
 {\BY}\@pfstepnum{1}{1} ,\, SMT {\DEF} SafeAt ,\, Phase1a ,\, VotedForIn ,\,
 WontVoteIn}%
 \@x{\makebox[0pt][r]{\scriptsize 174\hspace{1em}}\@pfstepnum{1}{2.}\ 
 {\ASSUME} {\NEW} a \.{\in} Acceptors ,\, Phase1b ( a )}%
 \@x{\makebox[0pt][r]{\scriptsize 175\hspace{1em}}\@s{23.88} {\PROVE}\@s{4.1}
 SafeAt ( v ,\, b ) \.{'}}%
 \@x{\makebox[0pt][r]{\scriptsize 176\hspace{1em}}\@s{8.2}
 {\BY}\@pfstepnum{1}{2} ,\, QuorumAssumption ,\, SMTT ( 60 ) {\DEF} TypeOK
 ,\, SafeAt ,\, WontVoteIn ,\, VotedForIn ,\, Phase1b}%
 \@x{\makebox[0pt][r]{\scriptsize 177\hspace{1em}}\@pfstepnum{1}{3.}\ 
 {\ASSUME} {\NEW} bb \.{\in} Ballots ,\, Phase2a ( bb )}%
 \@x{\makebox[0pt][r]{\scriptsize 178\hspace{1em}}\@s{23.88} {\PROVE}\@s{4.1}
 SafeAt ( v ,\, b ) \.{'}}%
 \@x{\makebox[0pt][r]{\scriptsize 179\hspace{1em}}\@s{8.2}
 {\BY}\@pfstepnum{1}{3} ,\, QuorumAssumption ,\, SMT {\DEF} TypeOK ,\, SafeAt
 ,\, WontVoteIn ,\, VotedForIn ,\, Phase2a}%
 \@x{\makebox[0pt][r]{\scriptsize 180\hspace{1em}}\@pfstepnum{1}{4.}\ 
 {\ASSUME} {\NEW} a \.{\in} Acceptors ,\, Phase2b ( a )}%
 \@x{\makebox[0pt][r]{\scriptsize 181\hspace{1em}}\@s{23.88} {\PROVE}\@s{4.1}
 SafeAt ( v ,\, b ) \.{'}}%
 \@x{\makebox[0pt][r]{\scriptsize 182\hspace{1em}}\@s{8.2}\@pfstepnum{2}{1.}\ 
 {\PICK} m \.{\in} msgs \.{:} Phase2b ( a ) {\bang} ( m )}%
 \@x{\makebox[0pt][r]{\scriptsize 183\hspace{1em}}\@s{16.4}
 {\BY}\@pfstepnum{1}{4}\  {\DEF} Phase2b}%
 \@x{\makebox[0pt][r]{\scriptsize 184\hspace{1em}}\@s{8.2}\@pfstepnum{2}{2}\ 
 \A\, aa \.{\in} Acceptors ,\, bb \.{\in} Ballots ,\, vv \.{\in} Values
 \.{:}}%
 \@x{\makebox[0pt][r]{\scriptsize 185\hspace{1em}}\@s{44.73} VotedForIn ( aa
 ,\, vv ,\, bb ) \.{\implies} VotedForIn ( aa ,\, vv ,\, bb ) \.{'}}%
 \@x{\makebox[0pt][r]{\scriptsize 186\hspace{1em}}\@s{16.4}
 {\BY}\@pfstepnum{2}{1}\  {\DEF} TypeOK ,\, VotedForIn}%
 \@x{\makebox[0pt][r]{\scriptsize 187\hspace{1em}}\@s{8.2}\@pfstepnum{2}{3.}\ 
 \A\, aa \.{\in} Acceptors ,\, bb \.{\in} Ballots \.{:} maxBal [ aa ] \.{>}
 bb \.{\implies} maxBal \.{'} [ aa ] \.{>} bb}%
 \@x{\makebox[0pt][r]{\scriptsize 188\hspace{1em}}\@s{16.4}
 {\BY}\@pfstepnum{2}{1}\  {\DEF} TypeOK}%
 \@x{\makebox[0pt][r]{\scriptsize 189\hspace{1em}}\@s{8.2}\@pfstepnum{2}{4.}\ 
 {\ASSUME} {\NEW} aa \.{\in} Acceptors ,\, {\NEW} bb \.{\in} Ballots ,\,}%
 \@x{\makebox[0pt][r]{\scriptsize 190\hspace{1em}}\@s{70.33} WontVoteIn ( aa
 ,\, bb ) ,\, {\NEW} vv \.{\in} Values ,\,}%
 \@x{\makebox[0pt][r]{\scriptsize 191\hspace{1em}}\@s{70.33} VotedForIn ( aa
 ,\, vv ,\, bb ) \.{'}}%
 \@x{\makebox[0pt][r]{\scriptsize 192\hspace{1em}}\@s{32.08} {\PROVE}
 {\FALSE}}%
 \@x{\makebox[0pt][r]{\scriptsize 193\hspace{1em}}\@s{16.4}\@pfstepnum{3}{}\ 
 {\DEFINE} mm \.{\defeq} [ type \.{\mapsto}\@w{2b} ,\, val \.{\mapsto} vv ,\,
 bal \.{\mapsto} bb ,\, acc \.{\mapsto} aa ]}%
 \@x{\makebox[0pt][r]{\scriptsize 194\hspace{1em}}\@s{16.4}\@pfstepnum{3}{1.}\
 mm \.{\notin} msgs}%
 \@x{\makebox[0pt][r]{\scriptsize 195\hspace{1em}}\@s{24.59}
 {\BY}\@pfstepnum{2}{4}\  {\DEF} WontVoteIn ,\, VotedForIn}%
 \@x{\makebox[0pt][r]{\scriptsize 196\hspace{1em}}\@s{16.4}\@pfstepnum{3}{2.}\
 mm \.{\in} msgs \.{'}}%
 \@x{\makebox[0pt][r]{\scriptsize
 197\hspace{1em}}\@s{24.59}\@pfstepnum{4}{1.}\  {\PICK} m1 \.{\in} msgs \.{'}
 \.{:}}%
 \@x{\makebox[0pt][r]{\scriptsize 198\hspace{1em}}\@s{56.68}
 \.{\land}\@s{5.33} m1 . type \.{=}\@w{2b}}%
 \@x{\makebox[0pt][r]{\scriptsize 199\hspace{1em}}\@s{56.68}
 \.{\land}\@s{5.33} m1 . val\@s{1.06} \.{=} vv}%
 \@x{\makebox[0pt][r]{\scriptsize 200\hspace{1em}}\@s{56.68}
 \.{\land}\@s{5.33} m1 . bal\@s{1.57} \.{=} bb}%
 \@x{\makebox[0pt][r]{\scriptsize 201\hspace{1em}}\@s{56.68}
 \.{\land}\@s{5.33} m1 . acc \.{=} aa}%
 \@x{\makebox[0pt][r]{\scriptsize 202\hspace{1em}}\@s{32.8}
 {\BY}\@pfstepnum{2}{4}\  {\DEF} VotedForIn}%
 \@x{\makebox[0pt][r]{\scriptsize 203\hspace{1em}}\@s{24.59}\@pfstepnum{4}{} .
 {\QED}\@s{4.1} {\BY}\@pfstepnum{4}{1}\  {\DEF} TypeOK ,\, Messages}%
\@y{\@s{0}%
 proved by \ensuremath{Zenon
}}%
\@xx{}%
 \@x{\makebox[0pt][r]{\scriptsize 204\hspace{1em}}\@s{16.4}\@pfstepnum{3}{3.}\
 aa \.{=} a \.{\land} m . bal \.{=} bb}%
 \@x{\makebox[0pt][r]{\scriptsize 205\hspace{1em}}\@s{24.59}
 {\BY}\@pfstepnum{2}{1} ,\,\@pfstepnum{3}{1} ,\,\@pfstepnum{3}{2}\  {\DEF}
 TypeOK}%
 \@x{\makebox[0pt][r]{\scriptsize 206\hspace{1em}}\@s{16.4}\@pfstepnum{3}{} .
 {\QED}}%
 \@x{\makebox[0pt][r]{\scriptsize 207\hspace{1em}}\@s{24.59}
 {\BY}\@pfstepnum{2}{1} ,\,\@pfstepnum{2}{4} ,\,\@pfstepnum{3}{3}\  {\DEF}
 Phase2b ,\, WontVoteIn ,\, TypeOK}%
 \@x{\makebox[0pt][r]{\scriptsize 208\hspace{1em}}\@s{8.2}\@pfstepnum{2}{5}\ 
 \A\, aa \.{\in} Acceptors ,\, bb \.{\in} Ballots \.{:} WontVoteIn ( aa ,\,
 bb ) \.{\implies} WontVoteIn ( aa ,\, bb ) \.{'}}%
 \@x{\makebox[0pt][r]{\scriptsize 209\hspace{1em}}\@s{16.4}
 {\BY}\@pfstepnum{2}{3} ,\,\@pfstepnum{2}{4}\  {\DEF} WontVoteIn}%
 \@x{\makebox[0pt][r]{\scriptsize 210\hspace{1em}}\@s{8.2}\@pfstepnum{2}{}\ 
 {\QED}}%
 \@x{\makebox[0pt][r]{\scriptsize 211\hspace{1em}}\@s{16.4}
 {\BY}\@pfstepnum{2}{2} ,\,\@pfstepnum{2}{5} ,\, QuorumAssumption {\DEF}
 SafeAt}%
\@pvspace{8.0pt}%
\@x{\makebox[0pt][r]{\scriptsize 213\hspace{1em}}\@pfstepnum{1}{5.}\  {\QED}}%
 \@x{\makebox[0pt][r]{\scriptsize 214\hspace{1em}}\@s{8.2}
 {\BY}\@pfstepnum{1}{1} ,\,\@pfstepnum{1}{2} ,\,\@pfstepnum{1}{3}
 ,\,\@pfstepnum{1}{4}\  {\DEF} Next}%
\@pvspace{8.0pt}%
 \@x{\makebox[0pt][r]{\scriptsize 216\hspace{1em}} {\THEOREM} Invariant
 \.{\defeq} Spec \.{\implies} {\Box} Inv}%
 \@x{\makebox[0pt][r]{\scriptsize 217\hspace{1em}}\@pfstepnum{1}{}\  {\USE}
 {\DEF} Ballots}%
 \@x{\makebox[0pt][r]{\scriptsize 218\hspace{1em}}\@pfstepnum{1}{1.}\  Init
 \.{\implies} Inv}%
 \@x{\makebox[0pt][r]{\scriptsize 219\hspace{1em}}\@s{8.2} {\BY} {\DEF} Init
 ,\, Inv ,\, TypeOK ,\, AccInv ,\, MsgInv ,\, VotedForIn}%
\@pvspace{8.0pt}%
 \@x{\makebox[0pt][r]{\scriptsize 221\hspace{1em}}\@pfstepnum{1}{2.}\  Inv
 \.{\land} [ Next ]_{ vars} \.{\implies} Inv \.{'}}%
 \@x{\makebox[0pt][r]{\scriptsize 222\hspace{1em}}\@s{8.2}\@pfstepnum{2}{}\ 
 {\SUFFICES} {\ASSUME} Inv ,\, Next}%
 \@x{\makebox[0pt][r]{\scriptsize 223\hspace{1em}}\@s{68.59} {\PROVE}\@s{4.1}
 Inv \.{'}}%
 \@x{\makebox[0pt][r]{\scriptsize 224\hspace{1em}}\@s{16.4} {\BY} {\DEF} vars
 ,\, Inv ,\, TypeOK ,\, MsgInv ,\, AccInv ,\, SafeAt ,\, VotedForIn ,\,
 WontVoteIn}%
 \@x{\makebox[0pt][r]{\scriptsize 225\hspace{1em}}\@s{8.2}\@pfstepnum{2}{}\ 
 {\USE} {\DEF} Inv}%
 \@x{\makebox[0pt][r]{\scriptsize 226\hspace{1em}}\@s{8.2}\@pfstepnum{2}{1.}\ 
 TypeOK \.{'}}%
 \@x{\makebox[0pt][r]{\scriptsize 227\hspace{1em}}\@s{16.4}\@pfstepnum{3}{1.}\
 {\ASSUME} {\NEW} b \.{\in} Ballots ,\, Phase1a ( b ) {\PROVE} TypeOK \.{'}}%
 \@x{\makebox[0pt][r]{\scriptsize 228\hspace{1em}}\@s{24.59}
 {\BY}\@pfstepnum{3}{1}\  {\DEF} TypeOK ,\, Phase1a ,\, Send ,\, Messages}%
 \@x{\makebox[0pt][r]{\scriptsize 229\hspace{1em}}\@s{16.4}\@pfstepnum{3}{2.}\
 {\ASSUME} {\NEW} b \.{\in} Ballots ,\, Phase2a ( b ) {\PROVE} TypeOK \.{'}}%
 \@x{\makebox[0pt][r]{\scriptsize
 230\hspace{1em}}\@s{24.59}\@pfstepnum{4}{1.}\  {\PICK} v \.{\in} Values
 \.{:}}%
 \@x{\makebox[0pt][r]{\scriptsize 231\hspace{1em}}\@s{60.78} \.{\land} Send (
 [ type \.{\mapsto}\@w{2a} ,\, bal \.{\mapsto} b ,\, val \.{\mapsto} v ] )}%
 \@x{\makebox[0pt][r]{\scriptsize 232\hspace{1em}}\@s{60.78} \.{\land}
 {\UNCHANGED} {\langle} maxBal ,\, maxVBal ,\, maxVal {\rangle}}%
 \@x{\makebox[0pt][r]{\scriptsize 233\hspace{1em}}\@s{32.8}
 {\BY}\@pfstepnum{3}{2}\  {\DEF} Phase2a}%
 \@x{\makebox[0pt][r]{\scriptsize 234\hspace{1em}}\@s{24.59}\@pfstepnum{4}{} .
 {\QED}}%
 \@x{\makebox[0pt][r]{\scriptsize 235\hspace{1em}}\@s{32.8}
 {\BY}\@pfstepnum{4}{1}\  {\DEF} TypeOK ,\, Send ,\, Messages}%
 \@x{\makebox[0pt][r]{\scriptsize 236\hspace{1em}}\@s{16.4}\@pfstepnum{3}{3.}\
 {\ASSUME} {\NEW} a \.{\in} Acceptors ,\, Phase1b ( a ) {\PROVE} TypeOK
 \.{'}}%
 \@x{\makebox[0pt][r]{\scriptsize 237\hspace{1em}}\@s{24.59}\@pfstepnum{4}{} .
 {\PICK} m \.{\in} msgs \.{:} Phase1b ( a ) {\bang} ( m )}%
 \@x{\makebox[0pt][r]{\scriptsize 238\hspace{1em}}\@s{32.8}
 {\BY}\@pfstepnum{3}{3}\  {\DEF} Phase1b}%
 \@x{\makebox[0pt][r]{\scriptsize 239\hspace{1em}}\@s{24.59}\@pfstepnum{4}{} .
 {\QED}}%
 \@x{\makebox[0pt][r]{\scriptsize 240\hspace{1em}}\@s{32.8} {\BY} {\DEF} Send
 ,\, TypeOK ,\, Messages}%
 \@x{\makebox[0pt][r]{\scriptsize 241\hspace{1em}}\@s{16.4}\@pfstepnum{3}{4.}\
 {\ASSUME} {\NEW} a \.{\in} Acceptors ,\, Phase2b ( a ) {\PROVE} TypeOK
 \.{'}}%
 \@x{\makebox[0pt][r]{\scriptsize 242\hspace{1em}}\@s{24.59}\@pfstepnum{4}{} .
 {\PICK} m \.{\in} msgs \.{:} Phase2b ( a ) {\bang} ( m )}%
 \@x{\makebox[0pt][r]{\scriptsize 243\hspace{1em}}\@s{32.8}
 {\BY}\@pfstepnum{3}{4}\  {\DEF} Phase2b}%
 \@x{\makebox[0pt][r]{\scriptsize 244\hspace{1em}}\@s{24.59}\@pfstepnum{4}{} .
 {\QED}}%
 \@x{\makebox[0pt][r]{\scriptsize 245\hspace{1em}}\@s{32.8} {\BY} {\DEF} Send
 ,\, TypeOK ,\, Messages}%
 \@x{\makebox[0pt][r]{\scriptsize 246\hspace{1em}}\@s{16.4}\@pfstepnum{3}{} .
 {\QED}}%
 \@x{\makebox[0pt][r]{\scriptsize 247\hspace{1em}}\@s{24.59}
 {\BY}\@pfstepnum{3}{1} ,\,\@pfstepnum{3}{2} ,\,\@pfstepnum{3}{3}
 ,\,\@pfstepnum{3}{4}\  {\DEF} Next}%
 \@x{\makebox[0pt][r]{\scriptsize 248\hspace{1em}}\@s{8.2}\@pfstepnum{2}{2.}\ 
 AccInv \.{'}}%
 \@x{\makebox[0pt][r]{\scriptsize 249\hspace{1em}}\@s{16.4}\@pfstepnum{3}{1.}\
 {\ASSUME} {\NEW} b \.{\in} Ballots ,\, Phase1a ( b ) {\PROVE} AccInv \.{'}}%
 \@x{\makebox[0pt][r]{\scriptsize 250\hspace{1em}}\@s{24.59}
 {\BY}\@pfstepnum{2}{1} ,\,\@pfstepnum{3}{1} ,\, SafeAtStable {\DEF} AccInv
 ,\, TypeOK ,\, Phase1a ,\, VotedForIn ,\, Send}%
 \@x{\makebox[0pt][r]{\scriptsize 251\hspace{1em}}\@s{16.4}\@pfstepnum{3}{2.}\
 {\ASSUME} {\NEW} b \.{\in} Ballots ,\, Phase2a ( b ) {\PROVE} AccInv \.{'}}%
 \@x{\makebox[0pt][r]{\scriptsize 252\hspace{1em}}\@s{32.8}
 {\BY}\@pfstepnum{2}{1} ,\,\@pfstepnum{3}{2} ,\, SafeAtStable {\DEF} AccInv
 ,\, TypeOK ,\, Phase2a ,\, VotedForIn ,\, Send}%
 \@x{\makebox[0pt][r]{\scriptsize 253\hspace{1em}}\@s{16.4}\@pfstepnum{3}{3.}\
 {\ASSUME} {\NEW} a \.{\in} Acceptors ,\, Phase1b ( a ) {\PROVE} AccInv
 \.{'}}%
 \@x{\makebox[0pt][r]{\scriptsize 254\hspace{1em}}\@s{32.8}
 {\BY}\@pfstepnum{2}{1} ,\,\@pfstepnum{3}{3} ,\, SafeAtStable {\DEF} AccInv
 ,\, TypeOK ,\, Phase1b ,\, VotedForIn ,\, Send}%
 \@x{\makebox[0pt][r]{\scriptsize 255\hspace{1em}}\@s{16.4}\@pfstepnum{3}{4.}\
 {\ASSUME} {\NEW} a \.{\in} Acceptors ,\, Phase2b ( a ) {\PROVE} AccInv
 \.{'}}%
 \@x{\makebox[0pt][r]{\scriptsize
 256\hspace{1em}}\@s{24.59}\@pfstepnum{4}{1.}\  {\PICK} m \.{\in} msgs \.{:}
 Phase2b ( a ) {\bang} ( m )}%
 \@x{\makebox[0pt][r]{\scriptsize 257\hspace{1em}}\@s{32.8}
 {\BY}\@pfstepnum{3}{4}\  {\DEF} Phase2b}%
 \@x{\makebox[0pt][r]{\scriptsize
 258\hspace{1em}}\@s{24.59}\@pfstepnum{4}{2.}\  \A\, acc \.{\in} Acceptors
 \.{:}}%
 \@x{\makebox[0pt][r]{\scriptsize 259\hspace{1em}}\@s{55.71} \.{\land} maxVal
 \.{'} [ acc ] \.{=} None \.{\equiv} maxVBal \.{'} [ acc ] \.{=} \.{-} 1}%
 \@x{\makebox[0pt][r]{\scriptsize 260\hspace{1em}}\@s{55.71} \.{\land} maxVBal
 \.{'} [ acc ] \.{\leq} maxBal \.{'} [ acc ]}%
 \@x{\makebox[0pt][r]{\scriptsize 261\hspace{1em}}\@s{32.8}
 {\BY}\@pfstepnum{2}{1} ,\,\@pfstepnum{4}{1} ,\, NoneNotAValue {\DEF} AccInv
 ,\, TypeOK ,\, Messages}%
 \@x{\makebox[0pt][r]{\scriptsize
 262\hspace{1em}}\@s{24.59}\@pfstepnum{4}{3.}\  \A\, aa ,\, vv ,\, bb \.{:}
 VotedForIn ( aa ,\, vv ,\, bb ) \.{'} \.{\equiv}}%
 \@x{\makebox[0pt][r]{\scriptsize 263\hspace{1em}}\@s{107.36} VotedForIn ( aa
 ,\, vv ,\, bb ) \.{\lor} ( aa \.{=} a \.{\land} vv \.{=} maxVal \.{'} [ a ]
 \.{\land} bb \.{=} maxVBal \.{'} [ a ] )}%
 \@x{\makebox[0pt][r]{\scriptsize 264\hspace{1em}}\@s{32.8}
 {\BY}\@pfstepnum{4}{1} ,\, Isa {\DEF} VotedForIn ,\, Send ,\, TypeOK ,\,
 Messages}%
 \@x{\makebox[0pt][r]{\scriptsize
 265\hspace{1em}}\@s{24.59}\@pfstepnum{4}{4.}\  {\ASSUME} {\NEW} acc \.{\in}
 Acceptors ,\, maxVBal \.{'} [ acc ] \.{\geq} 0}%
 \@x{\makebox[0pt][r]{\scriptsize 266\hspace{1em}}\@s{48.48} {\PROVE}\@s{4.1}
 VotedForIn ( acc ,\, maxVal [ acc ] ,\, maxVBal [ acc ] ) \.{'}}%
 \@x{\makebox[0pt][r]{\scriptsize 267\hspace{1em}}\@s{32.8}
 {\BY}\@pfstepnum{4}{1} ,\,\@pfstepnum{4}{3} ,\,\@pfstepnum{4}{4}\  {\DEF}
 AccInv ,\, TypeOK}%
 \@x{\makebox[0pt][r]{\scriptsize
 268\hspace{1em}}\@s{24.59}\@pfstepnum{4}{5.}\  {\ASSUME} {\NEW} acc \.{\in}
 Acceptors ,\, {\NEW} c \.{\in} Ballots ,\, c \.{>} maxVBal \.{'} [ acc ]
 ,\,}%
 \@x{\makebox[0pt][r]{\scriptsize 269\hspace{1em}}\@s{86.73} {\NEW} v \.{\in}
 Values ,\, VotedForIn ( acc ,\, v ,\, c ) \.{'}}%
 \@x{\makebox[0pt][r]{\scriptsize 270\hspace{1em}}\@s{48.48} {\PROVE}\@s{4.1}
 {\FALSE}}%
 \@x{\makebox[0pt][r]{\scriptsize 271\hspace{1em}}\@s{32.8}
 {\BY}\@pfstepnum{4}{1} ,\,\@pfstepnum{4}{3} ,\,\@pfstepnum{4}{5}
 ,\,\@pfstepnum{2}{1}\  {\DEF} AccInv ,\, TypeOK}%
 \@x{\makebox[0pt][r]{\scriptsize 272\hspace{1em}}\@s{24.59}\@pfstepnum{4}{} .
 {\QED}\@s{4.1} {\BY}\@pfstepnum{4}{2} ,\,\@pfstepnum{4}{4}
 ,\,\@pfstepnum{4}{5}\  {\DEF} AccInv}%
 \@x{\makebox[0pt][r]{\scriptsize 273\hspace{1em}}\@s{16.4}\@pfstepnum{3}{} .
 {\QED}}%
 \@x{\makebox[0pt][r]{\scriptsize 274\hspace{1em}}\@s{24.59}
 {\BY}\@pfstepnum{3}{1} ,\,\@pfstepnum{3}{2} ,\,\@pfstepnum{3}{3}
 ,\,\@pfstepnum{3}{4}\  {\DEF} Next}%
 \@x{\makebox[0pt][r]{\scriptsize 275\hspace{1em}}\@s{8.2}\@pfstepnum{2}{3.}\ 
 MsgInv \.{'}}%
 \@x{\makebox[0pt][r]{\scriptsize 276\hspace{1em}}\@s{16.4}\@pfstepnum{3}{1.}\
 {\ASSUME} {\NEW} b \.{\in} Ballots ,\, Phase1a ( b )}%
 \@x{\makebox[0pt][r]{\scriptsize 277\hspace{1em}}\@s{40.28} {\PROVE}\@s{4.1}
 MsgInv \.{'}}%
 \@x{\makebox[0pt][r]{\scriptsize
 278\hspace{1em}}\@s{24.59}\@pfstepnum{4}{1.}\  \A\, aa ,\, vv ,\, bb \.{:}
 VotedForIn ( aa ,\, vv ,\, bb ) \.{'} \.{\equiv} VotedForIn ( aa ,\, vv ,\,
 bb )}%
 \@x{\makebox[0pt][r]{\scriptsize 279\hspace{1em}}\@s{32.8}
 {\BY}\@pfstepnum{3}{1}\  {\DEF} Phase1a ,\, Send ,\, VotedForIn}%
 \@x{\makebox[0pt][r]{\scriptsize 280\hspace{1em}}\@s{24.59}\@pfstepnum{4}{} .
 {\QED}}%
 \@x{\makebox[0pt][r]{\scriptsize 281\hspace{1em}}\@s{32.8}
 {\BY}\@pfstepnum{3}{1} ,\,\@pfstepnum{4}{1} ,\, SafeAtStable
 ,\,\@pfstepnum{2}{1}\  {\DEF} Phase1a ,\, MsgInv ,\, TypeOK ,\, Messages ,\,
 Send}%
 \@x{\makebox[0pt][r]{\scriptsize 282\hspace{1em}}\@s{16.4}\@pfstepnum{3}{2.}\
 {\ASSUME} {\NEW} a \.{\in} Acceptors ,\, Phase1b ( a )}%
 \@x{\makebox[0pt][r]{\scriptsize 283\hspace{1em}}\@s{40.28} {\PROVE}\@s{4.1}
 MsgInv \.{'}}%
 \@x{\makebox[0pt][r]{\scriptsize 284\hspace{1em}}\@s{24.59}\@pfstepnum{4}{} .
 {\PICK} m \.{\in} msgs \.{:} Phase1b ( a ) {\bang} ( m )}%
 \@x{\makebox[0pt][r]{\scriptsize 285\hspace{1em}}\@s{32.8}
 {\BY}\@pfstepnum{3}{2}\  {\DEF} Phase1b}%
 \@x{\makebox[0pt][r]{\scriptsize
 286\hspace{1em}}\@s{24.59}\@pfstepnum{4}{1.}\  \A\, aa ,\, vv ,\, bb \.{:}
 VotedForIn ( aa ,\, vv ,\, bb ) \.{'} \.{\equiv} VotedForIn ( aa ,\, vv ,\,
 bb )}%
 \@x{\makebox[0pt][r]{\scriptsize 287\hspace{1em}}\@s{32.8} {\BY} {\DEF} Send
 ,\, VotedForIn}%
 \@x{\makebox[0pt][r]{\scriptsize 288\hspace{1em}}\@s{24.59}\@pfstepnum{4}{} .
 {\DEFINE} mm \.{\defeq} [ type \.{\mapsto}\@w{1b} ,\, bal \.{\mapsto} m .
 bal ,\, maxVBal \.{\mapsto} maxVBal [ a ] ,\,}%
 \@x{\makebox[0pt][r]{\scriptsize 289\hspace{1em}}\@s{114.63} maxVal
 \.{\mapsto} maxVal [ a ] ,\, acc \.{\mapsto} a ]}%
 \@x{\makebox[0pt][r]{\scriptsize
 290\hspace{1em}}\@s{24.59}\@pfstepnum{4}{2.}\  mm . bal \.{\leq} maxBal
 \.{'} [ mm . acc ]}%
 \@x{\makebox[0pt][r]{\scriptsize 291\hspace{1em}}\@s{32.8} {\BY} {\DEF}
 TypeOK ,\, Messages}%
 \@x{\makebox[0pt][r]{\scriptsize 292\hspace{1em}}\@s{24.59}\@pfstepnum{4}{3.}
 \.{\lor} \.{\land} mm . maxVal \.{\in} Values}%
 \@x{\makebox[0pt][r]{\scriptsize 293\hspace{1em}}\@s{56.26} \.{\land} mm .
 maxVBal \.{\in} Ballots}%
 \@x{\makebox[0pt][r]{\scriptsize 294\hspace{1em}}\@s{56.26} \.{\land}
 VotedForIn ( mm . acc ,\, mm . maxVal ,\, mm . maxVBal )}%
 \@x{\makebox[0pt][r]{\scriptsize 295\hspace{1em}}\@s{45.15} \.{\lor}
 \.{\land} mm . maxVal \.{=} None}%
 \@x{\makebox[0pt][r]{\scriptsize 296\hspace{1em}}\@s{56.26} \.{\land} mm .
 maxVBal \.{=} \.{-} 1}%
 \@x{\makebox[0pt][r]{\scriptsize 297\hspace{1em}}\@s{32.8} {\BY} {\DEF}
 TypeOK ,\, AccInv}%
 \@x{\makebox[0pt][r]{\scriptsize
 298\hspace{1em}}\@s{24.59}\@pfstepnum{4}{4.}\  \A\, c \.{\in} ( mm . maxVBal
 \.{+} 1 ) \.{\dotdot} ( mm . bal \.{-} 1 ) \.{:}}%
 \@x{\makebox[0pt][r]{\scriptsize 299\hspace{1em}}\@s{59.81} {\lnot} \E\, v
 \.{\in} Values \.{:} VotedForIn ( mm . acc ,\, v ,\, c )}%
 \@x{\makebox[0pt][r]{\scriptsize 300\hspace{1em}}\@s{32.8} {\BY} {\DEF}
 AccInv ,\, TypeOK ,\, Messages}%
 \@x{\makebox[0pt][r]{\scriptsize 301\hspace{1em}}\@s{24.59}\@pfstepnum{4}{} .
 {\QED}}%
 \@x{\makebox[0pt][r]{\scriptsize 302\hspace{1em}}\@s{32.8}
 {\BY}\@pfstepnum{4}{1} ,\,\@pfstepnum{4}{2} ,\,\@pfstepnum{4}{3}
 ,\,\@pfstepnum{4}{4} ,\, SafeAtStable {\DEF} MsgInv ,\, TypeOK ,\, Messages
 ,\, Send}%
 \@x{\makebox[0pt][r]{\scriptsize 303\hspace{1em}}\@s{16.4}\@pfstepnum{3}{3.}\
 {\ASSUME} {\NEW} b \.{\in} Ballots ,\, Phase2a ( b )}%
 \@x{\makebox[0pt][r]{\scriptsize 304\hspace{1em}}\@s{40.28} {\PROVE}\@s{4.1}
 MsgInv \.{'}}%
 \@x{\makebox[0pt][r]{\scriptsize
 305\hspace{1em}}\@s{24.59}\@pfstepnum{4}{1.}\  {\lnot} \E\, m \.{\in} msgs
 \.{:} ( m . type \.{=}\@w{2a} ) \.{\land} ( m . bal \.{=} b )}%
 \@x{\makebox[0pt][r]{\scriptsize 306\hspace{1em}}\@s{32.8}
 {\BY}\@pfstepnum{3}{3}\  {\DEF} Phase2a}%
 \@x{\makebox[0pt][r]{\scriptsize
 307\hspace{1em}}\@s{24.59}\@pfstepnum{4}{1a.}\  {\UNCHANGED} {\langle}
 maxBal ,\, maxVBal ,\, maxVal {\rangle}}%
 \@x{\makebox[0pt][r]{\scriptsize 308\hspace{1em}}\@s{32.8}
 {\BY}\@pfstepnum{3}{3}\  {\DEF} Phase2a}%
 \@x{\makebox[0pt][r]{\scriptsize
 309\hspace{1em}}\@s{24.59}\@pfstepnum{4}{2.}\  {\PICK} v \.{\in} Values
 \.{:}}%
 \@x{\makebox[0pt][r]{\scriptsize 310\hspace{1em}}\@s{60.78} \.{\land} \E\, Q
 \.{\in} Quorums \.{:}}%
 \@x{\makebox[0pt][r]{\scriptsize 311\hspace{1em}}\@s{79.12} \E\, S \.{\in}
 {\SUBSET} \{ m \.{\in} msgs \.{:} ( m . type \.{=}\@w{1b} ) \.{\land} ( m .
 bal \.{=} b ) \} \.{:}}%
 \@x{\makebox[0pt][r]{\scriptsize 312\hspace{1em}}\@s{86.34} \.{\land} \A\, a
 \.{\in} Q \.{:} \E\, m \.{\in} S \.{:} m . acc\@s{22.41} \.{=} a}%
 \@x{\makebox[0pt][r]{\scriptsize 313\hspace{1em}}\@s{86.34} \.{\land}
 \.{\lor} \A\, m \.{\in} S \.{:} m . maxVBal \.{=} \.{-} 1}%
 \@x{\makebox[0pt][r]{\scriptsize 314\hspace{1em}}\@s{97.45} \.{\lor} \E\,
 c\@s{3.77} \.{\in} 0 \.{\dotdot} ( b \.{-} 1 ) \.{:}}%
 \@x{\makebox[0pt][r]{\scriptsize 315\hspace{1em}}\@s{115.78} \.{\land} \A\, m
 \.{\in} S \.{:} m . maxVBal \.{\leq} c}%
 \@x{\makebox[0pt][r]{\scriptsize 316\hspace{1em}}\@s{115.78} \.{\land} \E\, m
 \.{\in} S \.{:} \.{\land} m . maxVBal \.{=} c}%
 \@x{\makebox[0pt][r]{\scriptsize 317\hspace{1em}}\@s{171.55} \.{\land} m .
 maxVal \.{=} v}%
 \@x{\makebox[0pt][r]{\scriptsize 318\hspace{1em}}\@s{60.78} \.{\land} Send (
 [ type \.{\mapsto}\@w{2a} ,\, bal \.{\mapsto} b ,\, val \.{\mapsto} v ] )}%
 \@x{\makebox[0pt][r]{\scriptsize 319\hspace{1em}}\@s{32.8}
 {\BY}\@pfstepnum{3}{3}\  {\DEF} Phase2a}%
 \@x{\makebox[0pt][r]{\scriptsize 320\hspace{1em}}\@s{24.59}\@pfstepnum{4}{} .
 {\DEFINE} mm \.{\defeq} [ type \.{\mapsto}\@w{2a} ,\, bal \.{\mapsto} b ,\,
 val \.{\mapsto} v ]}%
 \@x{\makebox[0pt][r]{\scriptsize
 321\hspace{1em}}\@s{24.59}\@pfstepnum{4}{3.}\  msgs \.{'} \.{=} msgs
 \.{\cup} \{ mm \}}%
 \@x{\makebox[0pt][r]{\scriptsize 322\hspace{1em}}\@s{32.8}
 {\BY}\@pfstepnum{4}{2}\  {\DEF} Send}%
 \@x{\makebox[0pt][r]{\scriptsize
 323\hspace{1em}}\@s{24.59}\@pfstepnum{4}{4.}\  \A\, aa ,\, vv ,\, bb \.{:}
 VotedForIn ( aa ,\, vv ,\, bb ) \.{'} \.{\equiv} VotedForIn ( aa ,\, vv ,\,
 bb )}%
 \@x{\makebox[0pt][r]{\scriptsize 324\hspace{1em}}\@s{32.8}
 {\BY}\@pfstepnum{4}{3}\  {\DEF} VotedForIn}%
 \@x{\makebox[0pt][r]{\scriptsize
 325\hspace{1em}}\@s{24.59}\@pfstepnum{4}{6.}\  \A\, m ,\, ma \.{\in} msgs
 \.{'} \.{:} m . type \.{=}\@w{2a} \.{\land} ma . type \.{=}\@w{2a} \.{\land}
 ma . bal \.{=} m . bal}%
 \@x{\makebox[0pt][r]{\scriptsize 326\hspace{1em}}\@s{131.07} \.{\implies} ma
 \.{=} m}%
 \@x{\makebox[0pt][r]{\scriptsize 327\hspace{1em}}\@s{32.8}
 {\BY}\@pfstepnum{4}{1} ,\,\@pfstepnum{4}{3} ,\, Isa {\DEF} MsgInv}%
 \@x{\makebox[0pt][r]{\scriptsize
 328\hspace{1em}}\@s{24.59}\@pfstepnum{4}{10.}\  SafeAt ( v ,\, b )}%
 \@x{\makebox[0pt][r]{\scriptsize 329\hspace{1em}}\@s{32.8}\@pfstepnum{5}{0.}\
 {\PICK} Q \.{\in} Quorums ,\,}%
 \@x{\makebox[0pt][r]{\scriptsize 330\hspace{1em}}\@s{81.33} S\@s{1.78}
 \.{\in} {\SUBSET} \{ m \.{\in} msgs \.{:} ( m . type \.{=}\@w{1b} )
 \.{\land} ( m . bal \.{=} b ) \} \.{:}}%
 \@x{\makebox[0pt][r]{\scriptsize 331\hspace{1em}}\@s{89.94} \.{\land} \A\, a
 \.{\in} Q \.{:} \E\, m \.{\in} S \.{:} m . acc \.{=} a}%
 \@x{\makebox[0pt][r]{\scriptsize 332\hspace{1em}}\@s{89.94} \.{\land}
 \.{\lor} \A\, m \.{\in} S \.{:} m . maxVBal \.{=} \.{-} 1}%
 \@x{\makebox[0pt][r]{\scriptsize 333\hspace{1em}}\@s{101.05} \.{\lor} \E\,
 c\@s{3.77} \.{\in} 0 \.{\dotdot} ( b \.{-} 1 ) \.{:}}%
 \@x{\makebox[0pt][r]{\scriptsize 334\hspace{1em}}\@s{119.38} \.{\land} \A\, m
 \.{\in} S \.{:} m . maxVBal \.{\leq} c}%
 \@x{\makebox[0pt][r]{\scriptsize 335\hspace{1em}}\@s{119.38} \.{\land} \E\, m
 \.{\in} S \.{:} \.{\land} m . maxVBal \.{=} c}%
 \@x{\makebox[0pt][r]{\scriptsize 336\hspace{1em}}\@s{175.14} \.{\land} m .
 maxVal \.{=} v}%
 \@x{\makebox[0pt][r]{\scriptsize 337\hspace{1em}}\@s{41.0}
 {\BY}\@pfstepnum{4}{2} ,\, Zenon}%
 \@x{\makebox[0pt][r]{\scriptsize 338\hspace{1em}}\@s{32.8}\@pfstepnum{5}{1.}
 {\CASE} \A\, m \.{\in} S \.{:} m . maxVBal \.{=} \.{-} 1}%
\@x{\makebox[0pt][r]{\scriptsize 339\hspace{1em}}\@s{41.0}}%
\@y{\@s{0}%
 In that case, no acceptor in \ensuremath{Q} voted in any ballot less than
 \ensuremath{b},
}%
\@xx{}%
\@x{\makebox[0pt][r]{\scriptsize 340\hspace{1em}}\@s{41.0}}%
\@y{\@s{0}%
 by the last conjunct of \ensuremath{MsgInv} for type \ensuremath{\@w{1b}}
 messages, and that\mbox{'}s enough
}%
\@xx{}%
 \@x{\makebox[0pt][r]{\scriptsize 341\hspace{1em}}\@s{41.0}
 {\BY}\@pfstepnum{5}{1} ,\,\@pfstepnum{5}{0}\  {\DEF} TypeOK ,\, MsgInv ,\,
 SafeAt ,\, WontVoteIn}%
 \@x{\makebox[0pt][r]{\scriptsize 342\hspace{1em}}\@s{32.8}\@pfstepnum{5}{2.}\
 {\ASSUME} {\NEW} c \.{\in} 0 \.{\dotdot} ( b \.{-} 1 ) ,\,}%
 \@x{\makebox[0pt][r]{\scriptsize 343\hspace{1em}}\@s{94.93} \A\, m \.{\in} S
 \.{:} m . maxVBal \.{\leq} c ,\,}%
 \@x{\makebox[0pt][r]{\scriptsize 344\hspace{1em}}\@s{94.93} {\NEW} ma \.{\in}
 S ,\, ma . maxVBal \.{=} c ,\, ma . maxVal \.{=} v}%
 \@x{\makebox[0pt][r]{\scriptsize 345\hspace{1em}}\@s{56.68} {\PROVE}\@s{4.1}
 SafeAt ( v ,\, b )}%
 \@x{\makebox[0pt][r]{\scriptsize 346\hspace{1em}}\@s{41.0}\@pfstepnum{6}{} .
 {\SUFFICES} {\ASSUME} {\NEW} d \.{\in} 0 \.{\dotdot} ( b \.{-} 1 )}%
 \@x{\makebox[0pt][r]{\scriptsize 347\hspace{1em}}\@s{100.84} {\PROVE}\@s{4.1}
 \E\, QQ\@s{6.49} \.{\in} Quorums \.{:} \A\, q \.{\in} QQ \.{:}}%
 \@x{\makebox[0pt][r]{\scriptsize 348\hspace{1em}}\@s{146.31} VotedForIn ( q
 ,\, v ,\, d ) \.{\lor} WontVoteIn ( q ,\, d )}%
 \@x{\makebox[0pt][r]{\scriptsize 349\hspace{1em}}\@s{49.2} {\BY} {\DEF}
 SafeAt}%
 \@x{\makebox[0pt][r]{\scriptsize 350\hspace{1em}}\@s{41.0}\@pfstepnum{6}{1.}
 {\CASE} d \.{\in} 0 \.{\dotdot} ( c \.{-} 1 )}%
\@x{\makebox[0pt][r]{\scriptsize 351\hspace{1em}}\@s{49.2}}%
\@y{\@s{0}%
 The \ensuremath{\@w{1b}} message for \ensuremath{v} with
 \ensuremath{maxVBal} value \ensuremath{c} must have been safe
}%
\@xx{}%
\@x{\makebox[0pt][r]{\scriptsize 352\hspace{1em}}\@s{49.2}}%
\@y{\@s{0}%
 according to \ensuremath{MsgInv} for \ensuremath{\@w{1b}} messages and lemma
 \ensuremath{VotedInv},
}%
\@xx{}%
\@x{\makebox[0pt][r]{\scriptsize 353\hspace{1em}}\@s{49.2}}%
\@y{\@s{0}%
 and that proves the assertion
}%
\@xx{}%
 \@x{\makebox[0pt][r]{\scriptsize 354\hspace{1em}}\@s{49.2}
 {\BY}\@pfstepnum{5}{2} ,\,\@pfstepnum{6}{1} ,\, VotedInv {\DEF} SafeAt ,\,
 MsgInv ,\, TypeOK ,\, Messages}%
 \@x{\makebox[0pt][r]{\scriptsize 355\hspace{1em}}\@s{41.0}\@pfstepnum{6}{2.}
 {\CASE} d \.{=} c}%
 \@x{\makebox[0pt][r]{\scriptsize 356\hspace{1em}}\@s{49.2}\@pfstepnum{7}{1.}\
 VotedForIn ( ma . acc ,\, v ,\, c )}%
 \@x{\makebox[0pt][r]{\scriptsize 357\hspace{1em}}\@s{57.4}
 {\BY}\@pfstepnum{5}{2}\  {\DEF} MsgInv}%
 \@x{\makebox[0pt][r]{\scriptsize 358\hspace{1em}}\@s{49.2}\@pfstepnum{7}{2.}\
 \A\, q \.{\in} Q ,\, w \.{\in} Values \.{:} VotedForIn ( q ,\, w ,\, c )
 \.{\implies} w \.{=} v}%
 \@x{\makebox[0pt][r]{\scriptsize 359\hspace{1em}}\@s{57.4}
 {\BY}\@pfstepnum{7}{1} ,\, VotedOnce ,\, QuorumAssumption {\DEF} TypeOK ,\,
 Messages}%
 \@x{\makebox[0pt][r]{\scriptsize 360\hspace{1em}}\@s{49.2}\@pfstepnum{7}{3.}\
 \A\, q \.{\in} Q \.{:} maxBal [ q ] \.{>} c}%
 \@x{\makebox[0pt][r]{\scriptsize 361\hspace{1em}}\@s{57.4}
 {\BY}\@pfstepnum{5}{0}\  {\DEF} MsgInv ,\, TypeOK ,\, Messages}%
 \@x{\makebox[0pt][r]{\scriptsize 362\hspace{1em}}\@s{49.2}\@pfstepnum{7}{} .
 {\QED}}%
 \@x{\makebox[0pt][r]{\scriptsize 363\hspace{1em}}\@s{57.4}
 {\BY}\@pfstepnum{6}{2} ,\,\@pfstepnum{7}{2} ,\,\@pfstepnum{7}{3}\  {\DEF}
 WontVoteIn}%
 \@x{\makebox[0pt][r]{\scriptsize 364\hspace{1em}}\@s{41.0}\@pfstepnum{6}{3.}
 {\CASE} d \.{\in} ( c \.{+} 1 ) \.{\dotdot} ( b \.{-} 1 )}%
\@x{\makebox[0pt][r]{\scriptsize 365\hspace{1em}}\@s{49.2}}%
\@y{\@s{0}%
 By the last conjunct of \ensuremath{MsgInv} for type \ensuremath{\@w{1b}}
 messages, no acceptor in \ensuremath{Q
}}%
\@xx{}%
\@x{\makebox[0pt][r]{\scriptsize 366\hspace{1em}}\@s{49.2}}%
\@y{\@s{0}%
 voted at any of these ballots.
}%
\@xx{}%
 \@x{\makebox[0pt][r]{\scriptsize 367\hspace{1em}}\@s{49.2}
 {\BY}\@pfstepnum{6}{3} ,\,\@pfstepnum{5}{0} ,\,\@pfstepnum{5}{2}\  {\DEF}
 MsgInv ,\, TypeOK ,\, Messages ,\, WontVoteIn}%
 \@x{\makebox[0pt][r]{\scriptsize 368\hspace{1em}}\@s{41.0}\@pfstepnum{6}{} .
 {\QED}\@s{4.1} {\BY}\@pfstepnum{6}{1} ,\,\@pfstepnum{6}{2}
 ,\,\@pfstepnum{6}{3}\ }%
 \@x{\makebox[0pt][r]{\scriptsize 369\hspace{1em}}\@s{32.8}\@pfstepnum{5}{} .
 {\QED}\@s{4.1} {\BY}\@pfstepnum{5}{0} ,\,\@pfstepnum{5}{1}
 ,\,\@pfstepnum{5}{2}\ }%
 \@x{\makebox[0pt][r]{\scriptsize
 370\hspace{1em}}\@s{24.59}\@pfstepnum{4}{11.}\  SafeAt ( mm . val ,\, mm .
 bal ) \.{'}}%
 \@x{\makebox[0pt][r]{\scriptsize 371\hspace{1em}}\@s{32.8}
 {\BY}\@pfstepnum{4}{10} ,\,\@pfstepnum{2}{1} ,\, SafeAtStable}%
 \@x{\makebox[0pt][r]{\scriptsize 372\hspace{1em}}\@s{24.59}\@pfstepnum{4}{} .
 {\QED}}%
\@x{\makebox[0pt][r]{\scriptsize 373\hspace{1em}}}%
\@y{\@s{17.5}%
 This proof used to work.
}%
\@xx{}%
 \@x{\makebox[0pt][r]{\scriptsize 374\hspace{1em}}\@s{36.89}
 {\BY}\@pfstepnum{2}{1} ,\,\@pfstepnum{4}{1a} ,\,\@pfstepnum{4}{3}
 ,\,\@pfstepnum{4}{4} ,\,\@pfstepnum{4}{6} ,\,\@pfstepnum{4}{11} ,\,
 SafeAtStable ,\, Zenon}%
 \@x{\makebox[0pt][r]{\scriptsize 375\hspace{1em}}\@s{45.1} {\DEF} MsgInv ,\,
 TypeOK ,\, Messages}%
\@x{\makebox[0pt][r]{\scriptsize 376\hspace{1em}}}%
\@y{\@s{17.5}%
 The following decomposition added by \ensuremath{LL} on 21 \ensuremath{Nov}
 2014 because
}%
\@xx{}%
\@x{\makebox[0pt][r]{\scriptsize 377\hspace{1em}}}%
\@y{\@s{17.5}%
 \ensuremath{Zenon} failed on this proof. However, \ensuremath{ZenonT(200)}
 worked.
}%
\@xx{}%
\begin{lcom}{17.5}%
\begin{cpar}{0}{T}{F}{2.5}{0}{}%
 \ensuremath{\@pfstepnum{5}{}} \ensuremath{{\SUFFICES}} \ensuremath{{\ASSUME}}
 \ensuremath{{\NEW}} \ensuremath{m \.{\in} msgs\.{'}
}%
\end{cpar}%
\begin{cpar}{0}{T}{F}{30.0}{0}{}%
\ensuremath{{\PROVE}} \ensuremath{MsgInv{\bang}(m)\.{'}
}%
\end{cpar}%
\begin{cpar}{1}{T}{F}{2.5}{0}{}%
\ensuremath{{\BY}} \ensuremath{{\DEF}} \ensuremath{MsgInv
}%
\end{cpar}%
\vshade{5.0}%
\begin{cpar}{2}{F}{F}{0}{0}{}%
\ensuremath{\@pfstepnum{5}{1.}} \ensuremath{m.type \.{=} \@w{1b}
}%
\end{cpar}%
\begin{cpar}{0}{T}{F}{17.5}{0}{}%
\ensuremath{\.{\implies}} (\ensuremath{\.{\land} m.bal \.{\leq} maxBal[m.acc]
}%
\end{cpar}%
\begin{cpar}{0}{T}{F}{10.0}{0}{}%
\ensuremath{\.{\land} \.{\lor} \.{\land} m.maxVal \.{\in} Values
}%
\end{cpar}%
\begin{cpar}{0}{T}{F}{15.0}{0}{}%
\ensuremath{\.{\land} m.maxVBal \.{\in} Nat
}%
\end{cpar}%
\begin{cpar}{0}{F}{F}{0}{0}{}%
\ensuremath{\.{\land} VotedForIn(m.acc,\, m.maxVal,\, m.maxVBal)
}%
\end{cpar}%
\begin{cpar}{1}{T}{F}{7.5}{0}{}%
\ensuremath{\.{\lor} \.{\land} m.maxVal \.{=} None
}%
\end{cpar}%
\begin{cpar}{0}{T}{F}{7.5}{0}{}%
\ensuremath{\.{\land} m.maxVBal \.{=} \.{-}1
}%
\end{cpar}%
\begin{cpar}{2}{F}{F}{0}{0}{}%
 \ensuremath{\.{\land} \A\, c \.{\in} m.maxVBal \.{+} 1\.{\dotdot}m.bal \.{-}
 1} :
\end{cpar}%
\begin{cpar}{0}{T}{F}{15.0}{0}{}%
 \ensuremath{{\lnot}(\E\, v\_1 \.{\in} Values \.{:} VotedForIn(m.acc,\,
 v\_1,\, c)))\.{'}
}%
\end{cpar}%
\begin{cpar}{3}{T}{F}{5.0}{5.0}{}%
 \ensuremath{{\BY}} \ensuremath{\@pfstepnum{2}{1}},
 \ensuremath{\@pfstepnum{4}{1a}}, \ensuremath{\@pfstepnum{4}{3}},
 \ensuremath{\@pfstepnum{4}{4}}, \ensuremath{\@pfstepnum{4}{6}},
 \ensuremath{\@pfstepnum{4}{11}}, \ensuremath{SafeAtStable
 \.{\,\backslash\,}}*, \ensuremath{Zenon
 } \ensuremath{{\DEF}} \ensuremath{MsgInv}, \ensuremath{TypeOK},
 \ensuremath{Messages
}%
\end{cpar}%
\vshade{5.0}%
\begin{cpar}{1}{F}{F}{0}{0}{}%
\ensuremath{\@pfstepnum{5}{2.}} \ensuremath{m.type \.{=} \@w{2a}
}%
\end{cpar}%
\begin{cpar}{0}{T}{F}{17.5}{0}{}%
\ensuremath{\.{\implies}} (\ensuremath{\.{\land} SafeAt(m.val,\, m.bal)
}%
\end{cpar}%
\begin{cpar}{0}{T}{F}{10.0}{0}{}%
\ensuremath{\.{\land} \A\, ma \.{\in} msgs} :
\end{cpar}%
\begin{cpar}{0}{T}{F}{15.0}{0}{}%
 \ensuremath{ma.type \.{=} \@w{2a} \.{\land} ma.bal \.{=} m.bal \.{\implies}
 ma \.{=} m)\.{'}
}%
\end{cpar}%
\begin{cpar}{3}{T}{F}{5.0}{5.0}{}%
 \ensuremath{{\BY}} \ensuremath{\@pfstepnum{2}{1}},
 \ensuremath{\@pfstepnum{4}{1a}}, \ensuremath{\@pfstepnum{4}{3}},
 \ensuremath{\@pfstepnum{4}{4}}, \ensuremath{\@pfstepnum{4}{6}},
 \ensuremath{\@pfstepnum{4}{11}}, \ensuremath{SafeAtStable
 \.{\,\backslash\,}}*, \ensuremath{Zenon
 } \ensuremath{{\DEF}} \ensuremath{MsgInv}, \ensuremath{TypeOK},
 \ensuremath{Messages
}%
\end{cpar}%
\vshade{5.0}%
\begin{cpar}{1}{F}{F}{0}{0}{}%
\ensuremath{\@pfstepnum{5}{3.}} \ensuremath{m.type \.{=} \@w{2b}
}%
\end{cpar}%
\begin{cpar}{0}{T}{F}{17.5}{0}{}%
\ensuremath{\.{\implies}} (\ensuremath{\.{\land} \E\, ma \.{\in} msgs} :
\end{cpar}%
\begin{cpar}{0}{T}{F}{25.0}{0}{}%
\ensuremath{\.{\land} ma.type \.{=} \@w{2a}
}%
\end{cpar}%
\begin{cpar}{0}{F}{F}{0}{0}{}%
\ensuremath{\.{\land} ma.bal \.{=} m.bal
}%
\end{cpar}%
\begin{cpar}{0}{F}{F}{0}{0}{}%
\ensuremath{\.{\land} ma.val \.{=} m.val
}%
\end{cpar}%
\begin{cpar}{1}{T}{F}{10.0}{0}{}%
\ensuremath{\.{\land} m.bal \.{\leq} maxVBal[m.acc])\.{'}
}%
\end{cpar}%
\begin{cpar}{2}{T}{F}{5.0}{5.0}{}%
 \ensuremath{{\BY}} \ensuremath{\@pfstepnum{2}{1}},
 \ensuremath{\@pfstepnum{4}{1a}}, \ensuremath{\@pfstepnum{4}{3}},
 \ensuremath{\@pfstepnum{4}{4}}, \ensuremath{\@pfstepnum{4}{6}},
 \ensuremath{\@pfstepnum{4}{11}},
 \ensuremath{SafeAtStable\.{\,\backslash\,}}* , \ensuremath{Zenon
 } \ensuremath{{\DEF}} \ensuremath{MsgInv}, \ensuremath{TypeOK},
 \ensuremath{Messages
}%
\end{cpar}%
\vshade{5.0}%
\begin{cpar}{1}{F}{F}{0}{0}{}%
\ensuremath{\@pfstepnum{5}{4.}} \ensuremath{{\QED}
}%
\end{cpar}%
\begin{cpar}{0}{T}{F}{5.0}{0}{}%
 \ensuremath{{\BY}} \ensuremath{\@pfstepnum{5}{1}},
 \ensuremath{\@pfstepnum{5}{2}}, \ensuremath{\@pfstepnum{5}{3}
}%
\end{cpar}%
\end{lcom}%
 \@x{\makebox[0pt][r]{\scriptsize 414\hspace{1em}}\@s{16.4}\@pfstepnum{3}{4.}\
 {\ASSUME} {\NEW} a \.{\in} Acceptors ,\, Phase2b ( a )}%
 \@x{\makebox[0pt][r]{\scriptsize 415\hspace{1em}}\@s{40.28} {\PROVE}\@s{4.1}
 MsgInv \.{'}}%
 \@x{\makebox[0pt][r]{\scriptsize 416\hspace{1em}}\@s{24.59}\@pfstepnum{4}{} .
 {\PICK} m \.{\in} msgs \.{:} Phase2b ( a ) {\bang} ( m )}%
 \@x{\makebox[0pt][r]{\scriptsize 417\hspace{1em}}\@s{32.8}
 {\BY}\@pfstepnum{3}{4}\  {\DEF} Phase2b}%
 \@x{\makebox[0pt][r]{\scriptsize
 418\hspace{1em}}\@s{24.59}\@pfstepnum{4}{1.}\  \A\, aa ,\, vv ,\, bb \.{:}
 VotedForIn ( aa ,\, vv ,\, bb ) \.{\implies} VotedForIn ( aa ,\, vv ,\, bb )
 \.{'}}%
 \@x{\makebox[0pt][r]{\scriptsize 419\hspace{1em}}\@s{32.8} {\BY} {\DEF}
 VotedForIn ,\, Send}%
 \@x{\makebox[0pt][r]{\scriptsize
 420\hspace{1em}}\@s{24.59}\@pfstepnum{4}{2.}\  \A\, mm \.{\in} msgs \.{:} mm
 . type \.{=}\@w{1b}}%
 \@x{\makebox[0pt][r]{\scriptsize 421\hspace{1em}}\@s{55.71}
 \.{\implies}\@s{1.56} \A\, v \.{\in} Values ,\, c\@s{14.75} \.{\in} ( mm .
 maxVBal \.{+} 1 ) \.{\dotdot} ( mm . bal \.{-} 1 ) \.{:}}%
 \@x{\makebox[0pt][r]{\scriptsize 422\hspace{1em}}\@s{80.05} {\lnot}
 VotedForIn ( mm . acc ,\, v ,\, c ) \.{\implies} {\lnot} VotedForIn ( mm .
 acc ,\, v ,\, c ) \.{'}}%
 \@x{\makebox[0pt][r]{\scriptsize 423\hspace{1em}}\@s{32.8} {\BY} {\DEF} Send
 ,\, VotedForIn ,\, MsgInv ,\, TypeOK ,\, Messages}%
 \@x{\makebox[0pt][r]{\scriptsize 424\hspace{1em}}\@s{24.59}\@pfstepnum{4}{} .
 {\QED}}%
 \@x{\makebox[0pt][r]{\scriptsize 425\hspace{1em}}\@s{32.8}
 {\BY}\@pfstepnum{4}{1} ,\,\@pfstepnum{4}{2} ,\, SafeAtStable
 ,\,\@pfstepnum{2}{1}\  {\DEF} MsgInv ,\, Send ,\, TypeOK ,\, Messages}%
 \@x{\makebox[0pt][r]{\scriptsize 426\hspace{1em}}\@s{16.4}\@pfstepnum{3}{5.}\
 {\QED}}%
 \@x{\makebox[0pt][r]{\scriptsize 427\hspace{1em}}\@s{24.59}
 {\BY}\@pfstepnum{3}{1} ,\,\@pfstepnum{3}{2} ,\,\@pfstepnum{3}{3}
 ,\,\@pfstepnum{3}{4}\  {\DEF} Next}%
 \@x{\makebox[0pt][r]{\scriptsize 428\hspace{1em}}\@s{8.2}\@pfstepnum{2}{4.}\ 
 {\QED}}%
 \@x{\makebox[0pt][r]{\scriptsize 429\hspace{1em}}\@s{16.4}
 {\BY}\@pfstepnum{2}{1} ,\,\@pfstepnum{2}{2} ,\,\@pfstepnum{2}{3}\  {\DEF}
 Inv}%
\@pvspace{8.0pt}%
\@x{\makebox[0pt][r]{\scriptsize 431\hspace{1em}}\@pfstepnum{1}{3.}\  {\QED}}%
 \@x{\makebox[0pt][r]{\scriptsize 432\hspace{1em}}\@s{8.2}
 {\BY}\@pfstepnum{1}{1} ,\,\@pfstepnum{1}{2} ,\, PTL {\DEF} Spec}%
\@pvspace{16.0pt}%
 \@x{\makebox[0pt][r]{\scriptsize 435\hspace{1em}} {\THEOREM} Consistent
 \.{\defeq} Spec \.{\implies} {\Box} Consistency}%
 \@x{\makebox[0pt][r]{\scriptsize 436\hspace{1em}}\@pfstepnum{1}{}\  {\USE}
 {\DEF} Ballots}%
\@pvspace{8.0pt}%
 \@x{\makebox[0pt][r]{\scriptsize 438\hspace{1em}}\@pfstepnum{1}{1.}\  Inv
 \.{\implies} Consistency}%
 \@x{\makebox[0pt][r]{\scriptsize 439\hspace{1em}}\@s{8.2}\@pfstepnum{2}{}\ 
 {\SUFFICES} {\ASSUME} Inv ,\,}%
 \@x{\makebox[0pt][r]{\scriptsize 440\hspace{1em}}\@s{106.84} {\NEW} v1
 \.{\in} Values ,\,\@s{4.1} {\NEW} v2 \.{\in} Values ,\,}%
 \@x{\makebox[0pt][r]{\scriptsize 441\hspace{1em}}\@s{106.84} {\NEW}
 b1\@s{0.44} \.{\in} Ballots ,\, {\NEW} b2\@s{2.63} \.{\in} Ballots ,\,}%
 \@x{\makebox[0pt][r]{\scriptsize 442\hspace{1em}}\@s{106.84} ChosenIn ( v1
 ,\, b1 ) ,\, ChosenIn ( v2 ,\, b2 ) ,\,}%
 \@x{\makebox[0pt][r]{\scriptsize 443\hspace{1em}}\@s{106.84} b1\@s{0.44}
 \.{\leq} b2}%
 \@x{\makebox[0pt][r]{\scriptsize 444\hspace{1em}}\@s{68.59} {\PROVE}\@s{4.1}
 v1 \.{=} v2}%
 \@x{\makebox[0pt][r]{\scriptsize 445\hspace{1em}}\@s{16.4} {\BY} {\DEF}
 Consistency ,\, Chosen}%
 \@x{\makebox[0pt][r]{\scriptsize 446\hspace{1em}}\@s{8.2}\@pfstepnum{2}{1.}
 {\CASE} b1 \.{=} b2}%
 \@x{\makebox[0pt][r]{\scriptsize 447\hspace{1em}}\@s{16.4}
 {\BY}\@pfstepnum{2}{1} ,\, VotedOnce ,\, QuorumAssumption ,\, SMTT ( 100 )
 {\DEF} ChosenIn ,\, Inv}%
\begin{lcom}{10.0}%
\begin{cpar}{0}{F}{F}{0}{0}{}%
 \ensuremath{\@pfstepnum{3}{1.}} \ensuremath{{\PICK}} \ensuremath{a1 \.{\in}
 Acceptors} : \ensuremath{VotedForIn(a1,\, v1,\, b1)
}%
\end{cpar}%
\begin{cpar}{0}{T}{F}{5.0}{0}{}%
 \ensuremath{{\BY}} \ensuremath{QuorumAssumption} \ensuremath{{\DEF}}
 \ensuremath{ChosenIn
}%
\end{cpar}%
\begin{cpar}{1}{F}{F}{0}{0}{}%
 \ensuremath{\@pfstepnum{3}{2.}} \ensuremath{{\PICK}} \ensuremath{a2 \.{\in}
 Acceptors} : \ensuremath{VotedForIn(a2,\, v2,\, b2)
}%
\end{cpar}%
\begin{cpar}{0}{T}{F}{5.0}{0}{}%
 \ensuremath{{\BY}} \ensuremath{QuorumAssumption} \ensuremath{{\DEF}}
 \ensuremath{ChosenIn
}%
\end{cpar}%
\begin{cpar}{1}{T}{F}{0}{5.0}{}%
\ensuremath{\@pfstepnum{3}{}}. \ensuremath{{\QED}
 } \ensuremath{{\BY}} \ensuremath{\@pfstepnum{3}{1}},
 \ensuremath{\@pfstepnum{3}{2}}, \ensuremath{\@pfstepnum{2}{1}},
 \ensuremath{VotedOnce} \ensuremath{{\DEF}} \ensuremath{Inv
}%
\end{cpar}%
\end{lcom}%
 \@x{\makebox[0pt][r]{\scriptsize 456\hspace{1em}}\@s{8.2}\@pfstepnum{2}{2.}
 {\CASE} b1 \.{<} b2}%
 \@x{\makebox[0pt][r]{\scriptsize 457\hspace{1em}}\@s{16.4}\@pfstepnum{3}{1.}\
 SafeAt ( v2 ,\, b2 )}%
 \@x{\makebox[0pt][r]{\scriptsize 458\hspace{1em}}\@s{24.59} {\BY} VotedInv
 ,\, QuorumNonEmpty ,\, QuorumAssumption {\DEF} ChosenIn ,\, Inv}%
 \@x{\makebox[0pt][r]{\scriptsize 459\hspace{1em}}\@s{16.4}\@pfstepnum{3}{2.}\
 {\PICK} Q2 \.{\in} Quorums \.{:}}%
 \@x{\makebox[0pt][r]{\scriptsize 460\hspace{1em}}\@s{78.54} \A\, a \.{\in}
 Q2\@s{14.10} \.{:} VotedForIn ( a ,\, v2 ,\, b1 ) \.{\lor} WontVoteIn ( a
 ,\, b1 )}%
 \@x{\makebox[0pt][r]{\scriptsize 461\hspace{1em}}\@s{24.59}
 {\BY}\@pfstepnum{3}{1} ,\,\@pfstepnum{2}{2}\  {\DEF} SafeAt}%
 \@x{\makebox[0pt][r]{\scriptsize 462\hspace{1em}}\@s{16.4}\@pfstepnum{3}{3.}\
 {\PICK} Q1 \.{\in} Quorums \.{:} \A\, a \.{\in} Q1 \.{:} VotedForIn ( a ,\,
 v1 ,\, b1 )}%
 \@x{\makebox[0pt][r]{\scriptsize 463\hspace{1em}}\@s{24.59} {\BY} {\DEF}
 ChosenIn}%
 \@x{\makebox[0pt][r]{\scriptsize 464\hspace{1em}}\@s{16.4}\@pfstepnum{3}{4.}\
 {\QED}}%
 \@x{\makebox[0pt][r]{\scriptsize 465\hspace{1em}}\@s{24.59}
 {\BY}\@pfstepnum{3}{2} ,\,\@pfstepnum{3}{3} ,\, QuorumAssumption ,\,
 VotedOnce ,\, Z3 {\DEF} WontVoteIn ,\, Inv}%
 \@x{\makebox[0pt][r]{\scriptsize 466\hspace{1em}}\@s{8.2}\@pfstepnum{2}{3.}\ 
 {\QED}}%
 \@x{\makebox[0pt][r]{\scriptsize 467\hspace{1em}}\@s{16.4}
 {\BY}\@pfstepnum{2}{1} ,\,\@pfstepnum{2}{2}\ }%
\@pvspace{8.0pt}%
\@x{\makebox[0pt][r]{\scriptsize 469\hspace{1em}}\@pfstepnum{1}{2.}\  {\QED}}%
 \@x{\makebox[0pt][r]{\scriptsize 470\hspace{1em}}\@s{8.2} {\BY} Invariant
 ,\,\@pfstepnum{1}{1} ,\, PTL}%
\@pvspace{8.0pt}%
\@x{\makebox[0pt][r]{\scriptsize 472\hspace{1em}}}\midbar\@xx{}%
 \@x{\makebox[0pt][r]{\scriptsize 473\hspace{1em}} chosenBar \.{\defeq} \{ v
 \.{\in} Values \.{:} Chosen ( v ) \}}%
\@pvspace{8.0pt}%
 \@x{\makebox[0pt][r]{\scriptsize 475\hspace{1em}} C \.{\defeq} {\INSTANCE}
 Consensus {\WITH} chosen \.{\leftarrow} chosenBar}%
\@pvspace{8.0pt}%
 \@x{\makebox[0pt][r]{\scriptsize 477\hspace{1em}} {\THEOREM} Refinement
 \.{\defeq} Spec \.{\implies} C {\bang} Spec}%
 \@x{\makebox[0pt][r]{\scriptsize 478\hspace{1em}}\@pfstepnum{1}{1.}\  Init
 \.{\implies} C {\bang} Init}%
 \@x{\makebox[0pt][r]{\scriptsize 479\hspace{1em}}\@s{8.2} {\BY}
 QuorumNonEmpty {\DEF} Init ,\, C {\bang} Init ,\, chosenBar ,\, Chosen ,\,
 ChosenIn ,\, VotedForIn}%
\@pvspace{8.0pt}%
 \@x{\makebox[0pt][r]{\scriptsize 481\hspace{1em}}\@pfstepnum{1}{2.}\  TypeOK
 \.{'} \.{\land} Consistency \.{'} \.{\land} [ Next ]_{ vars} \.{\implies} [
 C {\bang} Next ]_{ chosenBar}}%
 \@x{\makebox[0pt][r]{\scriptsize 482\hspace{1em}}\@s{8.2}\@pfstepnum{2}{}\ 
 {\SUFFICES} {\ASSUME} TypeOK \.{'} ,\, Consistency \.{'} ,\, Next ,\,
 chosenBar \.{'} \.{\neq} chosenBar}%
 \@x{\makebox[0pt][r]{\scriptsize 483\hspace{1em}}\@s{68.59} {\PROVE}\@s{4.1}
 C {\bang} Next}%
 \@x{\makebox[0pt][r]{\scriptsize 484\hspace{1em}}\@s{16.4} {\BY} {\DEF} vars
 ,\, chosenBar ,\, Chosen ,\, ChosenIn ,\, VotedForIn}%
 \@x{\makebox[0pt][r]{\scriptsize 485\hspace{1em}}\@s{8.2}\@pfstepnum{2}{1.}\ 
 chosenBar \.{\subseteq} chosenBar \.{'}}%
 \@x{\makebox[0pt][r]{\scriptsize 486\hspace{1em}}\@s{16.4} {\BY} {\DEF} Send
 ,\, chosenBar ,\, Chosen ,\, ChosenIn ,\, VotedForIn ,\, Next ,\, Phase1a
 ,\, Phase1b ,\, Phase2a ,\, Phase2b}%
 \@x{\makebox[0pt][r]{\scriptsize 487\hspace{1em}}\@s{8.2}\@pfstepnum{2}{2.}\ 
 \A\, v ,\, w \.{\in} chosenBar \.{'} \.{:} v \.{=} w}%
 \@x{\makebox[0pt][r]{\scriptsize 488\hspace{1em}}\@s{16.4} {\BY} {\DEF}
 Consistency ,\, chosenBar ,\, ChosenIn ,\, TypeOK}%
 \@x{\makebox[0pt][r]{\scriptsize 489\hspace{1em}}\@s{8.2}\@pfstepnum{2}{3.}\ 
 chosenBar \.{=} \{ \}}%
 \@x{\makebox[0pt][r]{\scriptsize 490\hspace{1em}}\@s{16.4}
 {\BY}\@pfstepnum{2}{1} ,\,\@pfstepnum{2}{2} ,\, SetExtensionality}%
 \@x{\makebox[0pt][r]{\scriptsize 491\hspace{1em}}\@s{8.2}\@pfstepnum{2}{} .
 {\QED}}%
 \@x{\makebox[0pt][r]{\scriptsize 492\hspace{1em}}\@s{16.4}
 {\BY}\@pfstepnum{2}{1} ,\,\@pfstepnum{2}{2} ,\,\@pfstepnum{2}{3}\  {\DEF} C
 {\bang} Next ,\, chosenBar}%
\@pvspace{8.0pt}%
\@x{\makebox[0pt][r]{\scriptsize 494\hspace{1em}}\@pfstepnum{1}{3.}\  {\QED}}%
 \@x{\makebox[0pt][r]{\scriptsize 495\hspace{1em}}\@s{8.2}
 {\BY}\@pfstepnum{1}{1} ,\,\@pfstepnum{1}{2} ,\, Invariant ,\, Consistent ,\,
 PTL {\DEF} Spec ,\, C {\bang} Spec ,\, Inv}%
\@x{\makebox[0pt][r]{\scriptsize 496\hspace{1em}}}\bottombar\@xx{}%
\setboolean{shading}{false}
\begin{lcom}{0}%
\begin{cpar}{0}{F}{F}{0}{0}{}%
\ensuremath{\.{\,\backslash\,}}* Modification History
\end{cpar}%
\begin{cpar}{0}{F}{F}{0}{0}{}%
 \ensuremath{\.{\,\backslash\,}}* Last modified \ensuremath{Mon}
 \ensuremath{Jul} 22 20:58:32 \ensuremath{CST} 2019 by \ensuremath{hengxin
}%
\end{cpar}%
\begin{cpar}{0}{F}{F}{0}{0}{}%
 \ensuremath{\.{\,\backslash\,}}* Last modified \ensuremath{Mon}
 \ensuremath{Jul} 22 20:30:39 \ensuremath{CST} 2019 by \ensuremath{hengxin
}%
\end{cpar}%
\begin{cpar}{0}{F}{F}{0}{0}{}%
 \ensuremath{\.{\,\backslash\,}}* Last modified \ensuremath{Fri}
 \ensuremath{Nov} 28 10:39:17 \ensuremath{PST} 2014 by \ensuremath{lamport
}%
\end{cpar}%
\begin{cpar}{0}{F}{F}{0}{0}{}%
 \ensuremath{\.{\,\backslash\,}}* Last modified Sun \ensuremath{Nov} 23
 14:45:09 \ensuremath{PST} 2014 by \ensuremath{lamport
}%
\end{cpar}%
\begin{cpar}{0}{F}{F}{0}{0}{}%
 \ensuremath{\.{\,\backslash\,}}* Last modified \ensuremath{Mon}
 \ensuremath{Nov} 24 02:03:02 \ensuremath{CET} 2014 by \ensuremath{merz
}%
\end{cpar}%
\begin{cpar}{0}{F}{F}{0}{0}{}%
 \ensuremath{\.{\,\backslash\,}}* Last modified Sat \ensuremath{Nov} 22
 12:04:19 \ensuremath{CET} 2014 by \ensuremath{merz
}%
\end{cpar}%
\begin{cpar}{0}{F}{F}{0}{0}{}%
 \ensuremath{\.{\,\backslash\,}}* Last modified \ensuremath{Fri}
 \ensuremath{Nov} 21 17:40:41 \ensuremath{PST} 2014 by \ensuremath{lamport
}%
\end{cpar}%
\begin{cpar}{0}{F}{F}{0}{0}{}%
 \ensuremath{\.{\,\backslash\,}}* Last modified \ensuremath{Tue}
 \ensuremath{Mar} 18 11:37:57 \ensuremath{CET} 2014 by \ensuremath{doligez
}%
\end{cpar}%
\begin{cpar}{0}{F}{F}{0}{0}{}%
 \ensuremath{\.{\,\backslash\,}}* Last modified Sat \ensuremath{Nov} 24
 18:53:09 \ensuremath{GMT\.{-}03}:00 2012 by \ensuremath{merz
}%
\end{cpar}%
\begin{cpar}{0}{F}{F}{0}{0}{}%
 \ensuremath{\.{\,\backslash\,}}* Created Sat \ensuremath{Nov} 17 16:02:06
 \ensuremath{PST} 2012 by \ensuremath{lamport
}%
\end{cpar}%
\end{lcom}%
\end{document}
